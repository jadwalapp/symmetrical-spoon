\documentclass[12pt,a4paper]{report}

% Packages
\usepackage[margin=1in,left=1.5in]{geometry}
\usepackage{times}
\usepackage{setspace}
\usepackage{titlesec}
\usepackage[nottoc,notlot,notlof]{tocbibind}
\usepackage{tocloft}
\usepackage{fancyhdr}
\usepackage{graphicx}
\usepackage{booktabs}
\usepackage{usecase}
\usepackage{xcolor}
\usepackage{hyperref}
\usepackage{enumitem}
\usepackage{indentfirst}
\usepackage{longtable}

\setlist[itemize]{leftmargin=1cm}
\setlist[enumerate]{leftmargin=1cm}

% hide links so no color appear, but they still work
\hypersetup{
    hidelinks
}

% Page numbering
\pagenumbering{roman}

% Title formatting
\titleformat{\chapter}{\normalfont\huge\bfseries\uppercase}{\thechapter}{20pt}{\huge}

% Document begin
\begin{document}

% Title Page
\begin{titlepage}
    \begin{center}
        \vspace*{1cm}
        \includegraphics[height=1cm]{images/yu-logo.png}\\[1cm]
        {\Large\bfseries AL YAMAMAH UNIVERSITY}\\[0.5cm]
        {\large College of Engineering and Architecture}\\[0.5cm]
        {\large Bachelor of Science in Software and Network Engineering}\\[2cm]
        {\Huge\bfseries 
            \begin{spacing}{1}
                Jadwal: An Elegant, iOS-based Calendar Manager
            \end{spacing}
        }
        \vspace{2cm}
        {\Large\bfseries Graduation Project}\\[2cm]
        \begin{center}
            \setlength{\fboxsep}{10pt}
            \setlength{\fboxrule}{1pt}
            \fbox{
                \begin{tabular}{p{0.45\linewidth}p{0.45\linewidth}}
                    \multicolumn{2}{c}{\textbf{Group Project Submission}} \\
                    \midrule
                    \textbf{Student Names} & \textbf{Student IDs} \\
                    \midrule
                    YAZED ALKHALAF & 202211123 \\
                    SAIMAN TAKLAS & 202021400 \\
                    AFFAN MOHAMMAD & 202211086 \\
                    ALI BA WAZIR & 202211018 \\
                    \midrule
                    \multicolumn{2}{l}{\textbf{Submission Date}: 19 Sep 2024} \\
                    \multicolumn{2}{l}{\textbf{Supervised By}: Dr. Inaya Allah} \\
                \end{tabular}
            }
        \end{center}
        \vfill
        {\large First Semester 2024--2025}
    \end{center}
\end{titlepage}

% Preface section
\chapter*{Abstract}
\addcontentsline{toc}{chapter}{Abstract}

In response to Saudi Arabia’s rapid modernization, time management has been a real challenge. This paper introduces Jadwal, that proposes an automatic schedule managing system where an individual can connect all the calendars and even extract the events that have been discussed on informal communication channels.
\\
\\
An iOS-based automatic schedule management system Jadwal, designed to enhance both personal and professional time organization, all-in-one scheduling experience. The primary objective is to offer a user-friendly interface that simplifies calendar management, helping users stay organized and maximize productivity. Through elegant design and intelligent event extraction from informal communication channels, Jadwal aspires to make seamless time management accessible for everyone.


\chapter*{Acknowledgment}
\addcontentsline{toc}{chapter}{Acknowledgment}

First and foremost, we would like to extend our heartfelt gratitude to our project members, whose dedication and hard work have been invaluable in achieving the initial milestone of our final project. Their commitment and teamwork have made it possible to help us reach the first stage on time, and we truly appreciate every effort contributed by them.\\
\\
Secondly, we would like to thank our Graduation project supervisor Dr. Inayath Ullah for his valuable guidance and precious time. His counsel was instrumental in shaping our project.
\\
\\
Lastly, I would like to thank all the professors and friends for their insightful recommendations which helped us to enhance our project and make it one of the most successful one. 

\newpage

\tableofcontents

\newpage
\listoffigures

\newpage

\chapter*{List of Abbreviations and Terminology}
\addcontentsline{toc}{chapter}{List of Abbreviations and Terminology}

\begin{center}
    \begin{longtable}{p{0.2\textwidth}p{0.7\textwidth}}
    \toprule
    \textbf{Term} & \textbf{Definition} \\
    \midrule
    \endhead
    
    \textbf{Calendar} & A list of events and dates within a year that are important to an organization or to the people involved in a particular activity. \cite{def:calendar} \\[1ex]
    
    \textbf{Jadwal} & Jadwal is a comprehensive time management tool designed to aggregate and optimize your existing calendars and data sources. \\[1ex]
    
    \textbf{WhatsApp} & WhatsApp is an alternative to SMS and offers simple, secure, reliable messaging and calling, available on phones all over the world. \cite{whatsapp-about} \\[1ex]
    
    \textbf{CalDAV} & Calendaring Extensions to WebDAV \\[1ex]
    
    \textbf{Golang} & Go is an open source project developed by a team at Google and many contributors from the open source community. \cite{def:Golang}\\[1ex]
    
    \textbf{JWT} & JSON Web Tokens are an open, industry standard RFC 7519 method for representing claims securely between two parties. \cite{def:JWT}\\[1ex]
    
    \textbf{WebDAV} & Web Distributed Authoring and Versioning \\[1ex]
    
    \textbf{HTTPS} & HyperText Transfer Protocol Secure \\[1ex]
    
    \textbf{iOS} & iPhone Operating System \\[1ex]
    
    \textbf{N/A} & Not Applicable \\[1ex]
    
    \textbf{UC} & Use Case \\[1ex]
    
    \textbf{ER diagram} & Entity Relationship ER Diagram is a type of flowchart that illustrates how entities \\[1ex]
    
    \textbf{Sequence Diagram} & An interaction diagram that details how operations are carried out. \\[1ex]
    
    \textbf{APIs} & Application programming interface. \\[1ex]
    
    \textbf{APNs} & Apple Push Notification service \\
    
    \bottomrule
    \end{longtable}
\end{center}

% Main body (switch to Arabic numerals)
\pagenumbering{arabic}

\chapter{Introduction}


\section{Background of the Project}

Calendars have been around a long time now, and they are a handy tool for humans. Both who are busy and who want to plan their days. People throughout history have used paper for calendars, but now with technology, things have changed. Calendars are digital now, and they can even be shared with others!

As the world is becoming one big village with globalization, people tend to squeeze every last minute of their days since competition is higher. Calendars help in that since they allow people to plan their days easily and keep track of when to meet people and do other activies.

\section{Problem Statement}

Keeping your calendar up to date with information is challenging, especially with the rise of many informal communcation channels like WhatsApp. People nowadays discuss when and where they will meet using those informal communication tools. This leads to calendars being out of sync from real life events you are committed to and might harm relations. The problem lies in the cumberness of adding events to a calendar manually, and sometimes out of busyness, you just forget that you didn't add the event to your calendar. Our Jadwal app aims to solve this issue for users in an elegant way that makes it seamless to manage your time confidently.

\section{Objectives of the Project}

The main objectives of Jadwal are:
\\
\begin{itemize}
    \item To develop an intelligent calendar management system that automatically extracts events from the informal communication channels like WhatsApp and adds them to the user's main calendar.
    \item To create a user friendly interface that allows users to easily add events to the calendar.
    \item To implement a smart resolution system that notifies users of scheduling conflicts and provides easy options for resolution.
    \item To integrate all the calendars into Jadwal's single calendar view to make viewing and managing all the events easy.
    \item To prioritize and automatically schedule daily routines such as waking time, sleeping time and prayer time.
    \item To significantly reduce the time users spend on manual calendar management.
\end{itemize}

\section{Scope of the Project}

Jadwal is not just another calendar application; it's a comprehensive time management tool designed to aggregate and optimize your existing calendars and data sources. The scope of the project includes:
\\
\begin{itemize}
    \item Development of an iOS application as the primary platform.
    \item Integration with calendars using CalDAV.
    \item WhatsApp message parsing for event extraction (subject to technical feasibility).
    \item Target audience: Busy professionals, students, and anyone juggling multiple schedules.
    \item User testing phase to ensure ease of use and effectiveness. Our testing methods will include:
       \\
    \begin{itemize}
            \item Beta testing with a diverse group of users.
            \item Analytics to track user behavior and app performance.
        \end{itemize}
\end{itemize}



\section{Significance of the Project}

Jadwal's significance can be summarized in the following points:
\\
\begin{enumerate}
    \item \textbf{Time is Money}: Since time is the only asset you can't get more of, Jadwal tries to make it less painful and less time consuming to have a good calendar throughout your day by parsing events from your informal communication channels like WhatsApp automatically.
    \\
    \item \textbf{Prayer First Calendar}: Prayer times come first, then your daily scheduled items.
    \\
    \item \textbf{Reduced Human Error}: Automated event extraction and addition to calendars minimize the risk of missing important events or appointments due to manual input errors or forgetfulness.
   \\
    \item \textbf{Conflict Resolution}: The smart resolution system helps users identify and resolve scheduling conflicts efficiently, reducing stress and improving overall time management.
   \\
    \item \textbf{Holistic View of Commitments}: By integrating multiple calendars into a single view, Jadwal provides users with a comprehensive overview of their commitments across various aspects of life, facilitating better decision-making and work-life balance.
\end{enumerate}

\section{Limitations of the Project}

Nothing is perfect, and our project is not an outlier. The limitations we have figured out about it are as follows:
\\
\begin{itemize}
    \item WhatsApp integration allows the app to read the users messages, so it would be hard to prove privacy hasn't been breached.
    \item WhatsApp integration might not always be there, they are a third-party.
    \item Learning new technologies for iOS development might require more time than anticipated.
    \item Accuracy of our algorithms to detect keywords indicating an event agreement has happened, especially for languages other than English.
    \item Time and manpower constraints may limit the number of features we can implement.
    \item Dependency on third-party APIs and their limitations.
\end{itemize}

\section{Organization of the Senior Project}

Our project plan can be illustrated in the following gantt chart, \textbf{Figure \ref{fig:project-gantt-chart}}.

\begin{figure}[!h]
    \centering
    \includegraphics[width=\textwidth]{images/gantt.png}
    \caption{Project Gantt Chart}
    \label{fig:project-gantt-chart}
\end{figure}

\chapter{Literature Review}

In developing Jadwal, we have drawn inspiration from and built upon existing research and products in the field of intelligent calendar management. Some key references include:
\\
\begin{itemize}
    \item \textbf{Clockwise (https://www.getclockwise.com/):} A smart calendar assistant that optimizes schedules and manages team coordination \cite{clockwise}. Clockwise's approach to intelligent time blocking and meeting optimization provides valuable insights for Jadwal's automated scheduling features.
    \\
    \item \textbf{Motion (https://www.usemotion.com/):} Motion's Intelligent Calendar takes your meetings, your tasks, your to-do list, your activities, and creates one perfect, optimized schedule to get it all done \cite{motion}.
   \\
    \item \textbf{Reclaim AI (https://reclaim.ai/):} An intelligent time management tool that helps optimize schedules and automate tasks \cite{reclaim}.
    \\
    \item \textbf{Calendi (https://calendi.ai/):} Calendi describes itself as: ``Calendi is an AI calendar system. Use it for scheduling tasks, automating meetings, and witness the future of calendar.'' \cite{calendi}
   \\
    \item \textbf{An Exploratory Study of Calendar Use:} ``Prospective remembering is the use of memory for remembering to do things in the future, as different from retrospective memory functions such as recalling past events.'' \cite{tungare2008exploratorystudycalendaruse}
    \\
    \item \textbf{WhatsApp Integration:} Our research indicates that direct WhatsApp integration for event extraction has not been widely implemented in existing calendar applications, making this a unique feature of Jadwal.
    \\
\end{itemize}

\begin{figure}[!h]
    \centering
    \includegraphics[width=\textwidth]{images/features-table.png}
    \caption{Feature Comparison Table}
    \label{fig:features-table}
\end{figure}

\newpage

In Figure \ref{fig:features-table} The table compares \textit{Jadwal}, our project, with four other scheduling applications—Clockwise, Motion, Reclaim AI, and Calendi—based based on several key features.
\\
\begin{itemize}
    \item \textbf{Open Source}: \textit{Jadwal} stands out as the only application that is open-source, allowing users and developers to access, modify, and improve the code. The other applications do not provide this.
    \\
    \item \textbf{WhatsApp Integration}: \textit{Jadwal} stands out again as the only application that connects and extract the event discussed over the platform by the user.
    \\
    \item \textbf{CalDAV Support}: \textit{Jadwal}, Clockwise, and Reclaim AI offer CalDAV support, which allows users to integrate calendars from different sources, while Motion and Calendi either does not sipport this.
    \\
    \item \textbf{Conflict Resolution}: All listed applications, including \textit{Jadwal}, supports conflict resolution, allowing users to manage overlapping events efficiently.
    \\
    \item \textbf{Prioritize Prayer Times}: \textit{Jadwal} is the only one offering this feature where the user be able to have prayer time scheduled
    \\
    \item \textbf{iOS Application}: \textit{Jadwal} is available on iOS, along with Clockwise, Reclaim AI, and Motion. The availability of Calendi on iOS is uncertain.
\end{itemize}

\chapter{System Analysis and Design}

\section{Functional Requirements}

\begin{itemize}
    \item The user shall be able to access their account using either Google OAuth or magic link via Email. For new users, a new accout is created, and for existing users, they are given access to their account directly.
    \item The system shall send a welcome email to new users.
    \item The user should be able to connect a calendar using CalDAV.
    \item The user should be able to connect their WhatsApp account.
    \item The user should be able to add events manually.
    \item The user should be able to view integrated calendar.
    \item The user should be able to manage scheduling conflicts.
    \item The user should be able to schedule prayer times.
    \item The system shall send event notifications to the user.
    \item The system shall add the WhatsApp extracted events to the calendar. If a conflict occurs, the user shall get a notification to resolve the conflict with suggestions.
    % todo: check if we should remove this or keep it before submission
    \item The system shall synchronize calendar data across multiple devices.
\end{itemize}

\section{Non-Functional Requirements}

\begin{itemize}
    \item \textbf{Platform Compatibility:} The app shall be compatible with iOS devices running iOS 16.0 or later.
    \item \textbf{Performance:} The app shall load the main calendar view within 3 seconds on 5G with speeds above 200mpbs.
    \item \textbf{User Experience:} The user interface shall follow iOS Human Interface Guidelines for consistency and ease of use.
    \item \textbf{Security:} All data transmissions between the app and servers shall be encrypted using HTTPS.
    \item \textbf{Localization:} The app shall support Arabic and English languages.
    % todo: check if we should remove this or keep it before submission
    \item \textbf{Data Privacy:} The app shall comply with the data protection regulations and laws in Saudi Arabia.
\end{itemize}
\newpage
\section{System use-cases}

                                                      
\textbf{Figure \ref{fig:use-case-diagram}} shows the use case diagram for the system of Jadwal.

\begin{figure}[!h]
    \centering
    \includegraphics[width=\textwidth]{images/use-case-diagram.png}
    \caption{Use Case Diagram of Jadwal}
    \label{fig:use-case-diagram}
\end{figure}

\begin{usecase}{Continue with Email}
    \ucbasicinfo{\#1}{HIGH}{Regular}
    \ucshortdescription{This UC allows users to login or create an account using their email.}
    \uctrigger{This UC starts when the user enters their email to the system.}
    \ucactors{User}{None}
    \ucpreconditions{User must have an email}
    \ucrelationships{Send Welcome Email}{N/A}{N/A}
    \ucinputsoutputs{
      \begin{itemize}
        \item \textbf{Email} (Source: User)
        \item \textbf{Magic Link (from email)} (Source: User)
      \end{itemize}
    }{
      \begin{itemize}
        \item \textbf{Magic link email} (Destination: User)
        \item \textbf{Confirmation messages} (Destination: User Interface)
      \end{itemize}
    }
    \ucmainflow{
      \begin{enumerate}
        \item The user enters their email.
          \ucinfo{System displays email input field.}
        \item System sends email and displays "Check your email" message.
          \ucinfo{WhatsApp sends the linking code.}
        \item User interacts with email client.
          \ucinfo{Confirmation of successful connection.}
        \item System verifies link and logs user in.
          \ucinfo{System processes link and updates user status.}
      \end{enumerate}
    }
    \ucalternateflows{
      \begin{enumerate}
        \item If the user has no account, the system creates a user.
      \end{enumerate}
    }
    \ucexceptions{
      \begin{itemize}
        \item Invalid email format.
        \item Email not registered (if login only).
        \item Magic link expired or invalid.
      \end{itemize}
    }
    \ucconclusion{This UC ends when the user is logged in.}
    \ucpostconditions{The system generates a JWT.}
    \ucspecialrequirements{An email server must be present to send email magic link.}
\end{usecase}
\begin{usecase}{Continue with Google}
  \ucbasicinfo{HIGH}{Regular}
  \ucshortdescription{This UC allows users to login or sign up with their Google account.}
  \uctrigger{This UC starts when the user clicks "Continue with Google" button in the app.}
  \ucactors{User}{Google}
  \ucpreconditions{The user must have an active Google account.}
  \ucrelationships{Send Welcome Email}{N/A}{N/A}
  \ucinputsoutputs{
    \begin{itemize}
      \item \textbf{Google access token} (Source: User)
    \end{itemize}
  }{
    \begin{itemize}
      \item \textbf{Authentication response} (Destination: User)
    \end{itemize}
  }
  \ucmainflow{
    \begin{enumerate}
      \item The user click continue with Google.
        \ucinfo{App uses OAuth to authenticate with Google}
      \item App sends Google access token to the system.
        \ucinfo{System verifies the token is issued for us and then issues JWT for usage within the app.}
    \end{enumerate}
  }
  \ucalternateflows{
    \begin{itemize}
      \item The user cancels the authentication request.
    \end{itemize}
  }
  \ucexceptions{
    \begin{itemize}
      \item Network issue
    \end{itemize}
  }
  \ucconclusion{This UC ends when the user is logged in.}
  \ucpostconditions{The system generates a JWT.}
\end{usecase}
\begin{usecase}{Send Welcome Email}
  \ucbasicinfo{Low}{Regular}
  \ucshortdescription{This UC welcomes the user to the platform.}
  \uctrigger{This UC starts when the user account is created.}
  \ucactors{User}{None}
  \ucpreconditions{User account must be created in the system.}
  \ucrelationships{N/A}{N/A}{N/A}
  \ucinputsoutputs{
    \begin{itemize}
      \item \textbf{User name} (Source: System)
      \item \textbf{Welcome email template} (Source: System)
    \end{itemize}
  }{
    \begin{itemize}
      \item \textbf{Welcome email} (Destination: User)
    \end{itemize}
  }
  \ucmainflow{
    \begin{enumerate}
      \item The system fetches the user information.
            \ucinfo{The database is used.}
      \item The system fetches the send welcome email template.
            \ucinfo{The template is filled with the user name.}
      \item The system sends the email with the template.
            \ucinfo{The email is received by the user welcoming them.}
    \end{enumerate}
  }
  \ucexceptions{
    \begin{itemize}
      \item Email server is down.
    \end{itemize}
  }
  \ucconclusion{This UC ends when the user receives an email from us welcoming them.}
  \ucspecialrequirements{An email server must be present to send welcome email.}
\end{usecase}

\begin{figure}[!h]
  \centering
  \includegraphics[width=\textwidth]{images/docs/diagrams/sequence-diagrams/all-sequence-diagrams/Send Welcome Email.png}
  \caption{Send Welcome Email Sequence Diagram}
  \label{fig:seq/send-welcome-email}
\end{figure}

The "sending a welcome email Sequence Diagram", shown in \textbf{Figure~\ref{fig:seq/send-welcome-email}}, shows the process of sending a welcome email to new customers. It begins with the system retrieving user data from the database using the FetchUserInfo function. Once the user data is fetched, the system requests a magic link email template via Get Magic Link Template. The template is then customized with the retrieved user data using Fill Magic Link Template by UserData. The system sends the filled email template to the email server via SendEmail, which delivers the email. Finally, the email server responds with EmailSentResponse, confirming the email's successful dispatch. This sequence ensures personalized and reliable email delivery.

\begin{usecase}{Log out}
  \ucbasicinfo{HIGH}{Regular}
  \ucshortdescription{User able to logout}
  \uctrigger{when the UC is done from the application}
  \ucactors{User}{}
  \ucpreconditions{User must log in}
  \ucrelationships{}{}{}
  \ucinputsoutputs{
    \begin{itemize}
      \item \textbf{Log-out} (Source: User)
    \end{itemize}
  }{
    \begin{itemize}
      \item \textbf{logged out from system} (Destination: System)
    \end{itemize}
  }
  \ucmainflow{
    \begin{enumerate}
      \item The user click the button Log-out
        \ucinfo{System displays "Log-out" button}
      \item Emphasize the Log-out.
        \ucinfo{The system may prompt the user to confirm the logout action}
      \item The user will be logged out from the system
        \ucinfo{Confirmation of successful Log-out}
    \end{enumerate}
  }
  \ucalternateflows{
    \begin{itemize}
      \item If the user selects "Cancel" on the confirmation prompt, the system returns to the previous page without logging out.
    \end{itemize}
  }
  \ucexceptions{
    \begin{itemize}
      \item System error
      \item Network issue
    \end{itemize}
  }
  \ucconclusion{}
  \ucpostconditions{The user will be logged out from the system}
  \ucbusinessrules{
    \begin{itemize}
      \item User should log in
    \end{itemize}
  }
  \ucspecialrequirements{Sign-in page should display clear button and error message}
\end{usecase}
\begin{usecase}{Connect Calendar}
  \ucbasicinfo{Medium}{Regular}
  \ucshortdescription{This UC allows the user to add Jadwal's Baikal calendar credentials using a .mobileconfig profile to their iOS device calendar accounts.}
  \uctrigger{This UC is triggered when the user wants clicks ``Easy Setup'' in the mobile app.}
  \ucactors{User}{iOS Settings}
  \ucpreconditions{User must be logged in}
  \ucrelationships{N/A}{N/A}{N/A}
  \ucinputsoutputs{
    \begin{itemize}
      \item \textbf{Magic Token with type CalDav} (Source: System)
      \item \textbf{.mobileconfig profile} (Source: System)
    \end{itemize}
  }{
    \begin{itemize}
      \item \textbf{Calendar configuration status} (Destination: iOS Settings)
    \end{itemize}
  }
  \ucmainflow{
    \begin{enumerate}
      \item The user taps "Easy Setup" in the app.
            \ucinfo{The app calls the backend to get a Magic Token of type CalDAV.}
      \item The app prompts the user to download the file from the backend endpoint.
            \ucinfo{The Magic Token is used to authenticate the user and get his credentials and give him his unique file to download.}
      \item The user approves the profile installation.
            \ucinfo{iOS configures the CalDAV account with our Baikal server automatically.}
      \item The app shows the user a success page when he goes back to it.
            \ucinfo{Success is shown only after confirming the calendar configuration is complete.}
    \end{enumerate}
  }
  \ucalternateflows{
    \begin{enumerate}
      \item If the user denies profile installation:
            \begin{itemize}
              \item The app shows an error page with "Setup cancelled - Try again later"
              \item The user can retry the setup process later
            \end{itemize}
      \item If the calendar configuration fails:
            \begin{itemize}
              \item The app shows an error page with "Calendar setup failed"
              \item The user may need to contact support or retry the process
            \end{itemize}
    \end{enumerate}
  }

  \ucconclusion{The UC ends when either the calendar is successfully configured or an error page is shown to the user.}
  \ucpostconditions{Either the user's iOS device is configured to sync with our Baikal calendar server, or the user is informed of the failure with appropriate guidance.}
  \ucspecialrequirements{The system must generate secure, user-specific .mobileconfig profiles.}
\end{usecase}

\begin{figure}[!h]
  \centering
  \includegraphics[width=\textwidth]{images/docs/diagrams/sequence-diagrams/all-sequence-diagrams/Connect Calendar.png}
  \caption{Connect Calendar Sequence Diagram}
  \label{fig:seq/connect-calendar}
\end{figure}

The ``Connect Calendar Sequence Diagram'', shown in \textbf{Figure~\ref{fig:seq/connect-calendar}}, illustrates the streamlined process of connecting an iOS device to Jadwal's Baikal calendar server. The sequence begins when the user taps ``Easy Setup'' in the app.

The process follows a secure flow where the app first requests a Magic Token of type CalDAV from the backend. Using this token, the app then requests a user-specific \textit{.mobileconfig} profile that contains the pre-configured CalDAV credentials. When the user downloads this profile, iOS Settings takes over to handle the secure installation process.

The diagram illustrates the following possible paths:
\begin{enumerate}
  \item \textbf{Success Path:} The user approves the profile installation in iOS Settings, which then automatically configures the CalDAV account. Upon returning to the app, the user is shown a success page after the system confirms the calendar configuration is complete.
  \item \textbf{User Denial Path:} If the user denies the profile installation, they return to the app which displays "Setup cancelled - Try again later", allowing them to retry the process at their convenience.
  \item \textbf{Configuration Failure Path:} If the profile installation is approved but the calendar configuration fails, the app shows a "Calendar setup failed" message, prompting the user to either contact support or retry the process.
\end{enumerate}

This implementation leverages iOS's native configuration profile system to ensure a secure and user-friendly setup process. The use of Magic Tokens and encrypted \textit{.mobileconfig} profiles during transit guarantees that calendar credentials are transmitted and stored securely on the user's device. The clear error handling paths ensure users receive appropriate guidance when issues occur during any stage of the setup process.
\begin{usecase}{Create Calendar}
  \ucbasicinfo{High}{Regular}
  \ucshortdescription{This UC allows the user to create a new calendar using iOS EventKit.}
  \uctrigger{This UC is triggered when the user clicks ``Create Calendar'' in the app.}
  \ucactors{User}{EventKit}
  \ucpreconditions{
    \begin{itemize}
      \item User must be logged in
      \item Calendar access must be authorized in iOS Settings
      \item Baikal CalDAV account must be configured
    \end{itemize}
  }
  \ucrelationships{N/A}{N/A}{N/A}
  \ucinputsoutputs{
    \begin{itemize}
      \item \textbf{Calendar name} (Source: User)
      \item \textbf{Calendar account to add calendar under} (Source: User)
      \item \textbf{Calendar color} (Source: User)
    \end{itemize}
  }{
    \begin{itemize}
      \item \textbf{New calendar} (Destination: EventKit)
      \item \textbf{Creation status} (Destination: App)
    \end{itemize}
  }
  \ucmainflow{
    \begin{enumerate}
      \item The user taps the ``Calendar'' icon button in the app toolbar in the ``Calendar'' screen.
            \ucinfo{The app presents a ``Calendars'' sheet.}
      \item The user clicks the ``plus'' icon button in the toolbar of the ``Calendars'' sheet.
            \ucinfo{The app presents a form for calendar name, selector to choose account to add calendar under, and calendar color.}
      \item The user enters calendar name, selects a color, and selector to choose account to add calender under.
            \ucinfo{The app uses EventKit to create a new calendar.}
      \item EventKit creates the calendar and handles synchronization.
            \ucinfo{The calendar appears in the user's calendar list.}
    \end{enumerate}
  }
  \ucalternateflows{
    \begin{itemize}
      \item If calendar creation fails, the app shows an error and allows retry.
    \end{itemize}
  }
  \ucexceptions{
    \begin{itemize}
      \item \textbf{EventKit access denied:} The app prompts to enable calendar access in Settings.
      \item \textbf{Network issues:} EventKit handles synchronization internally.
    \end{itemize}
  }
  \ucconclusion{The UC ends when EventKit confirms the calendar creation.}
  \ucpostconditions{A new calendar is created through EventKit.}
  \ucspecialrequirements{The system must use EventKit for all calendar operations.}
\end{usecase}

\begin{figure}[!h]
  \centering
  \includegraphics[width=0.5\textwidth]{images/docs/diagrams/sequence-diagrams/all-sequence-diagrams/Create Calendar.png}
  \caption{Create Calendar Sequence Diagram}
  \label{fig:seq/create-calendar}
\end{figure}

The ``Create Calendar Sequence Diagram'', shown in \textbf{Figure~\ref{fig:seq/create-calendar}}, illustrates the process of creating a new calendar in the Jadwal app. The sequence begins when the user accesses the calendar creation interface through the ``Calendars'' sheet, accessed via the calendar icon in the toolbar.

The flow involves several key steps:
\begin{enumerate}
  \item The user provides essential calendar details through a form interface:
        \begin{itemize}
          \item Calendar name
          \item Account selection (where to add the calendar)
          \item Calendar color
        \end{itemize}
  \item The app communicates with EventKit to create the new calendar
  \item EventKit handles the calendar creation and all necessary synchronization
\end{enumerate}

The process is designed to be robust and user-friendly:
\begin{itemize}
  \item If EventKit access is denied, the app guides users to enable calendar access in iOS Settings
  \item If creation fails, the app shows an error and allows retry
  \item EventKit handles all synchronization internally
\end{itemize}

This implementation leverages iOS's native EventKit framework, which manages all calendar operations and synchronization internally. This provides a reliable and native iOS experience while ensuring proper calendar management.
\begin{usecase}{Connect WhatsApp}
    \ucbasicinfo{\#7}{HIGH}{Regular}
    \ucshortdescription{This UC allows the user to connect their WhatsApp account to our system.}
    \uctrigger{This UC is triggered when the user clicks on "Connect WhatsApp" button in the app.}
    \ucactors{User}{WhatsApp}
    \ucpreconditions{User must be logged in}
    \ucrelationships{N/A}{N/A}{N/A}
    \ucinputsoutputs{
      \begin{itemize}
        \item \textbf{WhatsApp account number} (Source: User)
        \item \textbf{WhatsApp linking code} (Source: User)
      \end{itemize}
    }{
      \begin{itemize}
        \item \textbf{WhatsApp linking code} (Destination: WhatsApp)
        \item \textbf{WhatsApp auth creds} (Destination: System)
      \end{itemize}
    }
    \ucmainflow{
      \begin{enumerate}
        \item The user clicks "Connect WhatsApp" button.
        \ucinfo{System asks for phone number via a dialog}
        \item The user enters their WhatsApp account phone number.
        \ucinfo{WhatsApp sends the linking code.}
        \item The user enters the linking code shown in the WhatsApp app in our app.
        \ucinfo{Confirmation of successful connection}
      \end{enumerate}
    }
    \ucalternateflows{None}
    \ucexceptions{
      \begin{itemize}
        \item \textbf{Wrong linking code:} If the user enters a wrong linking code, the connection of the WhatsApp account will fail unless they enter the correct  code.
        \item \textbf{Network issue:} A network issue interrupting the communication between the app, the server, and WhatsApp.
      \end{itemize}
    }
    \ucpostconditions{The system has access to the user's WhatsApp account.}
    \ucspecialrequirements{None}
    \ucconclusion{The UC ends when the user has a connected WhatsApp account.}
\end{usecase}
\begin{usecase}{Extract Events from WhatsApp}
  \ucbasicinfo{High}{Regular}
  \ucshortdescription{Allows users to extract event details shared in WhatsApp messages and save them to their calendar.}
  \uctrigger{User selects a message containing event details in WhatsApp.}
  \ucactors{User}{WhatsApp}
  \ucpreconditions{User must have WhatsApp installed and linked to the system.}
  \ucrelationships{N/A}{Event Management}{N/A}
  \ucinputsoutputs{
    \begin{itemize}
      \item \textbf{User-selected WhatsApp message containing event details.} (Source: User)
    \end{itemize}
  }{
    \begin{itemize}
      \item \textbf{Extracted event details saved to the calendar.} (Destination: User)
    \end{itemize}
  }
  \ucmainflow{
    \begin{enumerate}
      \item User opens WhatsApp and selects a message with event details. 
        \ucinfo{Display of the selected message.}
      \item User taps on the ``Extract Event'' option.
        \ucinfo{System processes the message to identify event details.}
      \item System extracts event details from the message.
        \ucinfo{Date, time, and location are parsed from the text. }
    \end{enumerate}
  }
  \ucalternateflows{
    \begin{itemize}
      \item If the message does not contain valid event details, display an error message.
      \item Notify the user about invalid format. 
    \end{itemize}
  }
  \ucexceptions{
    \begin{itemize}
      \item If there is a system error during extraction, display a relevant error message. 
      \item Log error for troubleshooting.
    \end{itemize}
  }
  \ucconclusion{User successfully extracts event details from WhatsApp and saves them to the calendar.}
  \ucpostconditions{The event is added to the user's calendar, and the user can view it in their schedule.}
  \ucbusinessrules{
    \begin{itemize}
      \item Only messages with recognized event formats can be extracted.
    \end{itemize}
  }
  \ucspecialrequirements{The system must have permissions to access WhatsApp messages and extract relevant information.}
\end{usecase}
\begin{usecase}{Suggest Conflict Resolutions}
  \ucbasicinfo{High}{Regular}
  \ucshortdescription{This UC gives the user all the conflicts and possible ways to resolve it.}
  \uctrigger{The UC is triggered when a conflict is detected between any overlapping event}
  \ucactors{User}{WhatsApp}
  \ucpreconditions{The calendar must have events}
  \ucrelationships{N/A}{Manage Scheduling Conflicts}{N/A}
  \ucinputsoutputs{
    \begin{itemize}
      \item \textbf{Overlapping events} (Source: Saved Events in Database)
      \item \textbf{User suggested resolution} (Source: User)
    \end{itemize}
  }{
    \begin{itemize}
      \item \textbf{Resolution options} (Destination: User Interface)
      \item \textbf{Updated calendar schedule} (Destination: Calendar)
    \end{itemize}
  }
  \ucmainflow{
    \begin{enumerate}
      \item The system detection conflicts
            \ucinfo{The system detects overlapping of events either added manually or extracted from WhatsApp and gives a notification to the user.}
      \item The system gives the suggestions for Conflicts.
            \ucinfo{The system provides the user with the list of resolution options.
              \begin{itemize}
                \item By moving the overlapping event to another time slot.
                \item Keep both events with a conflict warning.
              \end{itemize}}
    \end{enumerate}
  }
  \ucalternateflows{
    \begin{enumerate}
      \item {No conflicts detected}
    \end{enumerate}
  }
  \ucexceptions{
    \begin{itemize}
      \item If the system doesn't get any possible way to resolve the conflict then the system would mark both the event as conflicting.
    \end{itemize}
  }
  \ucconclusion{The UC ends when the user chooses resolution weather if to reschedule the event or leaving it without resolving and it is reflected in on the calendar.}
  \ucpostconditions{The conflicting events in the calendar are either resolved or marked as conflicting}
  \ucspecialrequirements{The system give feasible conflict resolution options}
\end{usecase}

\begin{figure}[!h]
  \centering
  \includegraphics[width=\textwidth]{images/docs/diagrams/sequence-diagrams/all-sequence-diagrams/Suggest Conflict Resolutions.png}
  \caption{Suggest Conflict Resolutions Sequence Diagram}
  \label{fig:seq/suggest-conflict-resolutions}
\end{figure}

The sequence diagram in Figure~\ref{fig:seq/suggest-conflict-resolutions} illustrates the conflict detection and resolution workflow in Jadwal. When events are added to the calendar (either manually or through WhatsApp extraction), the System initiates a conflict check in the Database. This check specifically looks for temporal overlaps between events and identifies potential alternative time slots.

If conflicts are detected, the System follows a structured resolution process:
\begin{enumerate}
  \item Retrieves resolution options from the Database, which include:
        \begin{itemize}
          \item Moving the overlapping event to an alternative time slot
          \item Keeping both events with an explicit conflict warning
        \end{itemize}
  \item Obtains the device IDs associated with the customer who owns the conflicting events
  \item Utilizes Apple Push Notification service (APNs) to notify users about the conflict and present them with resolution options
\end{enumerate}

If no conflicts are detected, the System continues without any additional actions. This approach ensures users are promptly informed of scheduling conflicts while maintaining the flexibility to either reschedule events or knowingly maintain overlapping appointments. The workflow aligns with Jadwal's goal of providing straightforward conflict management while respecting user preferences in calendar organization.

This process specifically implements the use case requirements, focusing on practical conflict resolution through user notification and simple resolution options, rather than attempting automated resolution or complex prioritization schemes.
\begin{usecase}{Manage Scheduling Conflicts}
  \ucbasicinfo{Medium}{Regular}
  \ucshortdescription{This UC allows the user to manage scheduling conflicts by suggesting resolutions when overlapping events are detected.}
  \uctrigger{This UC is triggered when an automatically added event overlaps with an existing event.}
  \ucactors{User}{None}
  \ucpreconditions{
    \begin{itemize}
      \item User is logged in.
      \item Conflicting events list is not empty.
    \end{itemize}
  }
  \ucrelationships{N/A}{N/A}{N/A}
  \ucinputsoutputs{
    \begin{itemize}
      \item \textbf{Conflicting events} (Source: System)
      \item \textbf{Event to override} (Source: User)
    \end{itemize}
  }{
    \begin{itemize}
      \item \textbf{Conflict resolution suggestion} (Destination: User Interface)
      \item \textbf{Updated event schedules} (Destination: Calendar)
    \end{itemize}
  }
  \ucmainflow{
    \begin{enumerate}
      \item The user opens the application and clicks on the view conflicts icon
        \ucinfo{The application shows all the conflics with their resolution options}
      \item The user chooses the best fit option to manage each conflict 
        \ucinfo{The conflict is resolved and is removed from the conflict list }

    \end{enumerate}
  }
  \ucalternateflows{
    \begin{enumerate}
      \item If the user doesn't choose any option, it shows conflicting status until the user chooses any option or the event expires. 
      \item If the user clicks on reject the event is left overlapping.
    \end{enumerate}
  }
  \ucexceptions{
    \begin{itemize}
      \item Netwok failure 
    \end{itemize}
  }
  \ucconclusion{The UC ends when the conflicting events are either resolved or marked as conflicting, based on the user's choice.}
  \ucpostconditions{The calendar reflects the user's decision regarding event conflicts.}
  \ucspecialrequirements{The system should provide best suggestions for resolving conflicts.}
\end{usecase}

\begin{usecase}{Add Event Manually}
  \ucbasicinfo{High}{Regular}
  \ucshortdescription{This UC allows users to add events manually.}
  \uctrigger{The user clicks add event manually icon or a date on the calendar and adds the events.}
  \ucactors{User}{None}
  \ucpreconditions{The user is logged into the application.}
  \ucrelationships{Suggest Conflict Resolutions}{N/A}{N/A}
  \ucinputsoutputs{
    \begin{itemize}
      \item \textbf{Event Name} (Source: User)
      \item \textbf{Event Location} (Source: User)
      \item \textbf{Is all day?} (Source: User)
      \item \textbf{Event Date (Start and End)} (Source: User)
      \item \textbf{Event Time (Start and End)} (Source: User)
      \item \textbf{Event Description} (Source: User)
      \item \textbf{Notifications/Reminders} (Source: User)
    \end{itemize}
  }{
    \begin{itemize}
      \item \textbf{New Calendar event}
            (Destination: Calendar)
    \end{itemize}
  }
  \ucmainflow{
    \begin{enumerate}
      \item The user clicks the add event manually icon or a date on the calendar.
            \ucinfo{The add event manually form is displayed.}
      \item The user sets the details of the event in the respective fields and saves the event.
            \ucinfo{The event is displayed on the calendar with its details.}
    \end{enumerate}
  }
  \ucalternateflows{
    \begin{enumerate}
      \item If the validation fails the user can try again afte fixing the issues
    \end{enumerate}
  }
  \ucexceptions{
    \begin{itemize}
      \item The end time is before the start time.
      \item The user attempts to save the event without filling in mandatory fields.
    \end{itemize}
  }
  \ucconclusion{The UC ends when the event has been successfully added to the calendar, and displayed.}
  \ucpostconditions{The event is successfully added to the calendar and displayed in the correct time slot.}
  \ucspecialrequirements{The interface must be simple and allowing users to input events with less efforts.}
\end{usecase}

\begin{figure}[!h]
  \centering
  \includegraphics[width=\textwidth]{images/docs/diagrams/sequence-diagrams/all-sequence-diagrams/Add Event Manually.png}
  \caption{Add Event Manually Sequence Diagram}
  \label{fig:seq/add-event-manually}
\end{figure}

The "Add Event Manually Sequence Diagram", shown in \textbf{Figure~\ref{fig:seq/add-event-manually}}, illustrates the process flow when a user manually creates a new calendar event. The sequence begins when the user submits event details through the CreateEvent endpoint, triggering a series of validation and storage operations.

The Backend first performs comprehensive validation of the event details, checking for:
\begin{itemize}
  \item Mandatory fields completion (event name, date, time)
  \item Temporal logic (end time after start time)
  \item Format validity of all provided fields
\end{itemize}

Upon successful validation, the system executes the following steps:
\begin{enumerate}
  \item Stores the validated event in the Database
  \item Performs an automatic conflict check with existing events
  \item If conflicts are detected:
        \begin{itemize}
          \item Retrieves the device IDs associated with the event owner
          \item Dispatches a "New Conflict Detected" notification via Apple Push Notification service (APNs)
        \end{itemize}
  \item Returns an EventCreated response to the user
\end{enumerate}

If validation fails, the system immediately returns a ValidationError to the user, preventing invalid data from entering the system. This workflow ensures data integrity while providing immediate feedback about potential scheduling conflicts, maintaining calendar consistency and user awareness of overlapping appointments.
\begin{usecase}{View Integrated Calendar}
    \ucbasicinfo{High}{Regular}
    \ucshortdescription{This UC allows the user to view their calendar events from all configured calendars through iOS's native calendar system.}
    \uctrigger{This UC is triggered when the user opens the calendar view in the app.}
    \ucactors{User}{EventKit}
    \ucpreconditions{
        \begin{itemize}
            \item User must be logged in
            \item User must have granted calendar access permission
            \item User must have at least one calendar configured in iOS
        \end{itemize}
    }
    \ucrelationships{N/A}{N/A}{N/A}
    \ucinputsoutputs{
        \begin{itemize}
            \item \textbf{Calendar view preferences} (Source: User)
            \item \textbf{Calendar events} (Source: iOS Calendar System)
        \end{itemize}
    }{
        \begin{itemize}
            \item \textbf{Integrated calendar view} (Destination: App)
        \end{itemize}
    }
    \ucmainflow{
        \begin{enumerate}
            \item The user opens the calendar view in the app.
                  \ucinfo{The app uses EventKit to fetch events from all configured calendars.}
            \item The app displays the integrated calendar view.
                  \ucinfo{Events from all calendars are shown in a unified view.}
            \item The user can interact with the calendar view.
                  \ucinfo{The user can view event details, switch between different view modes (day, week, month), and filter calendars.}
        \end{enumerate}
    }
    \ucalternateflows{
        \begin{itemize}
            \item If no calendars are configured, the app shows an empty state with instructions to add calendars through iOS settings.
            \item If calendar access is revoked, the app prompts the user to grant access in iOS settings.
        \end{itemize}
    }
    \ucexceptions{
        \begin{itemize}
            \item \textbf{Calendar access denied:} If the app doesn't have calendar access permission.
            \item \textbf{Event fetch failure:} If EventKit fails to retrieve calendar events.
        \end{itemize}
    }
    \ucconclusion{The UC ends when the user has a clear view of their calendar events from all configured calendars.}
    \ucpostconditions{The user can view and interact with events from all their configured calendars in a unified interface.}
\end{usecase}

\begin{figure}[!h]
    \centering
    \includegraphics[width=0.5\textwidth]{images/docs/diagrams/sequence-diagrams/all-sequence-diagrams/View Integrated Calendar.png}
    \caption{View Integrated Calendar Sequence Diagram}
    \label{fig:seq/view-integrated-calendar}
\end{figure}

The ``View Integrated Calendar Sequence Diagram'', shown in \textbf{Figure~\ref{fig:seq/view-integrated-calendar}}, illustrates how the app uses EventKit to fetch and display calendar events from iOS's native calendar system.

The process involves:
\begin{enumerate}
    \item The app requests calendar events through EventKit when the user opens the calendar view
    \item EventKit fetches events from all configured calendars in iOS
    \item The app displays the events in an integrated view
    \item The user can interact with the events and switch between different view modes
\end{enumerate}

This approach leverages iOS's built-in calendar infrastructure to provide a seamless calendar viewing experience.
\begin{usecase}{Schedule Prayer Times}
    \ucbasicinfo{High}{Regular}
    \ucshortdescription{The calendar is blocked and updated automatically according to the person's time zone prayer time.}
    \uctrigger{This usecase triggered when the user enables the prayer time feature in the application.}
    \ucactors{User}{None}
    \ucpreconditions{User must be logged into the system.}
    \ucrelationships{N/A}{N/A}{N/A}
    \ucinputsoutputs{
      \begin{itemize}
        \item \textbf{User's time Zone} (Source: user)
        \item \textbf{We get the IP address of the user and get their time zone}
      \end{itemize}
    }{
      \begin{itemize}
        \item \textbf{The calendar displays the blocked time for prayer time according to their time zone.}
      \end{itemize}
      }
    \ucmainflow{
      \begin{enumerate}
        \item User enables the feature by clicking the the option Schedule Prayer Time.
          \ucinfo{The system checks the time zone of the user and blocks the calendar accordingly.}
      \end{enumerate}
    }
    \ucalternateflows{
      \begin{itemize}
        \item The user doesn't enable the scheduling prayer time.
      \end{itemize}
    }
    \ucexceptions{
      \begin{itemize}
        \item If there's a system error, display a relevant error message.
      \end{itemize}
    }
    \ucpostconditions{The system generates calendar with prayer time}
    \ucspecialrequirements{The system must block the calendar according to their time zone}
    \ucconclusion{User's prayer times are successfully scheduled.}
    \ucbusinessrules{
      \begin{itemize}
        \item Prayer times must be within valid time ranges.
      \end{itemize}
    }
\end{usecase}
\begin{usecase}{Receive Event Notifications}
  \ucbasicinfo{High}{Regular}
  \ucshortdescription{Users receive notifications about upcoming events.}
  \uctrigger{The UC is triggered when the choosen time of an event's reminder has arrived}
  \ucactors{User}{None}
  \ucpreconditions{User must be logged into the system and set a reminder of the specific event.}
  \ucrelationships{N/A}{N/A}{N/A}
  \ucinputsoutputs{
    \begin{itemize}
      \item \textbf{Time of the event reminder} (Source: User)
    \end{itemize}
  }{
    \begin{itemize}

      \item \textbf{Notifications sent to users.} (Destination: System)
    \end{itemize}
  }
  \ucmainflow{
    \begin{enumerate}
      \item The system checks the alarms set for every event.
            \ucinfo{The system checks every 1 minute for set alarms for every event.}
      \item Notifications sent to users.
            \ucinfo{The user is reminded about the event by sending the notification}
    \end{enumerate}
  }
  \ucalternateflows{
    \begin{itemize}
      \item \textbf{If user denys the access to the notification the notifications are not sent.}
    \end{itemize}
  }
  \ucexceptions{
    \begin{itemize}
      \item \textbf{Network issue}
    \end{itemize}
  }
  \ucconclusion{The system checks for set alarm every 1 minute and if the event is detected the system sends the notification.}
  \ucpostconditions{Notifications are sent.}
  \ucspecialrequirements{Notification permission.}
  \ucbusinessrules{The system has to check every minute for the set alarm.
  }
\end{usecase}

\begin{figure}[!h]
  \centering
  \includegraphics[width=\textwidth]{images/docs/diagrams/sequence-diagrams/all-sequence-diagrams/Receive Event Notifications.png}
  \caption{Receive Event Notifications Sequence Diagram}
  \label{fig:seq/receive-event-notifications}
\end{figure}

The ``Receive Event Notifications Sequence Diagram'', shown in \textbf{Figure~\ref{fig:seq/receive-event-notifications}}, illustrates Jadwal's continuous event notification monitoring and delivery system. The sequence operates in a continuous polling loop that executes every minute, ensuring timely notification delivery for all scheduled events.

The polling process consists of several key steps:
\begin{enumerate}
  \item The System queries the Database for active alarms through regular checks
  \item For each batch of active alarms found:
        \begin{itemize}
          \item Retrieves associated device IDs for notification targets
          \item Iterates through each event requiring notification
          \item Dispatches push notifications via Apple Push Notification service (APNs)
        \end{itemize}
  \item If no active alarms are found, the system continues its polling cycle
\end{enumerate}

This robust notification system ensures reliable delivery of event reminders while efficiently managing system resources. The one-minute polling interval provides a balance between timely notifications and system performance, while the batch processing of notifications optimizes the interaction with APNs. The system's ability to handle multiple device IDs per user ensures notifications reach users across all their registered devices.

\section{Database Design}

\begin{figure}[!h]
    \centering
    \includegraphics[width=0.9\textwidth]{images/docs/diagrams/er/database/Database Design.png}
    \caption{Database Design}
    \label{fig:database-design}
\end{figure}

% Bibliography
\bibliography{references}
\bibliographystyle{apalike}

\end{document}