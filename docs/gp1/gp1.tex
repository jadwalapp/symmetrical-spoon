\documentclass[12pt,a4paper]{report}

% Packages
\usepackage[margin=1in,left=1.5in]{geometry}
\usepackage{times}
\usepackage{setspace}
\usepackage{titlesec}
\usepackage[nottoc,notlot,notlof]{tocbibind}
\usepackage{tocloft}
\usepackage{fancyhdr}
\usepackage{graphicx}
\usepackage{booktabs}
\usepackage{usecase}
\usepackage{xcolor}
\usepackage{hyperref}

% hide links so no color appear, but they still work
\hypersetup{
    hidelinks
}

% Page numbering
\pagenumbering{roman}

% Title formatting
\titleformat{\chapter}{\normalfont\huge\bfseries\uppercase}{\thechapter}{20pt}{\huge}

% Document begin
\begin{document}

% Title Page
\begin{titlepage}
    \begin{center}
        \vspace*{1cm}
        \includegraphics[height=1cm]{images/yu-logo.png}\\[1cm]
        {\Large\bfseries AL YAMAMAH UNIVERSITY}\\[0.5cm]
        {\large College of Engineering and Architecture}\\[0.5cm]
        {\large Bachelor of Science in Software and Network Engineering}\\[2cm]
        {\Huge\bfseries 
            \begin{spacing}{1}
                Jadwal: An Elegant, iOS-based Calendar Manager
            \end{spacing}
        }
        \vspace{2cm}
        {\Large\bfseries Graduation Project}\\[2cm]
        \begin{center}
            \setlength{\fboxsep}{10pt}
            \setlength{\fboxrule}{1pt}
            \fbox{
                \begin{tabular}{p{0.45\linewidth}p{0.45\linewidth}}
                    \multicolumn{2}{c}{\textbf{Group Project Submission}} \\
                    \midrule
                    \textbf{Student Names} & \textbf{Student IDs} \\
                    \midrule
                    YAZED ALKHALAF & 202211123 \\
                    SAIMAN TAKLAS & 202021400 \\
                    AFFAN MOHAMMAD & 202211086 \\
                    ALI BA WAZIR & 202211018 \\
                    \midrule
                    \multicolumn{2}{l}{\textbf{Submission Date}: 19 Sep 2024} \\
                    \multicolumn{2}{l}{\textbf{Supervised By}: Dr. Inaya Allah} \\
                \end{tabular}
            }
        \end{center}
        \vfill
        {\large First Semester 2024--2025}
    \end{center}
\end{titlepage}

% Preface section
\chapter*{Abstract}
\addcontentsline{toc}{chapter}{Abstract}

\chapter*{Acknowledgment}
\addcontentsline{toc}{chapter}{Acknowledgment}
\newpage

\tableofcontents

\newpage
\listoffigures

\newpage
\listoftables
\newpage

\chapter*{List of Abbreviations}
\addcontentsline{toc}{chapter}{List of Abbreviations}

% Main body (switch to Arabic numerals)
\pagenumbering{arabic}

\chapter{Introduction}


\section{Background of the Study}

As the world is moving towards globalizing, effective time management is becoming very important. Considering how everything seems to be rushing in today's world, there is a high requirement for an effective user-friendly time management tool. Paper-based calendars have been used for addressing the complexities of managing multiple schedules across various aspects of life such as work, school, and personal commitments. However, they often fall short in providing a comprehensive solution to modern scheduling challenges.

The introduction of the digital calendar has somewhat solved this problem but still, users face a lot of issues in keeping their calendars up-to-date and synchronized. There are still few people that manually input events into their calendars. This could be really tiring, especially when dealing with multiple calendars.

Moreover, the rise of instant messaging platforms like WhatsApp has changed the way we communicate and plan events. Mostly, important dates and appointments are discussed informally leading to a disconnect between where the information is initially shared and where it needs to be recorded for effective time management.

To address these challenges, we are planning an application called \textit{Jadwal}, which aims to revolutionize how people manage their time and schedules in the digital age.

\section{Problem Statement}

Users often face challenges in keeping their calendars up-to-date, particularly when dealing with information from various sources, including informal communication mediums like WhatsApp. The process of manually adding events to the calendar is both time-consuming and prone to errors. Additionally, managing multiple calendars—such as those for work, school, and personal life—creates further complexity and increases the risk of scheduling conflicts. The lack of seamless integration with popular communication platforms exacerbates the problem, leading to a higher likelihood of missing important events due to the scattered distribution of information across different calendars and data sources.

\section{Objectives of the Study}

The main objectives of Jadwal are:

\begin{itemize}
    \item To develop an intelligent calendar management system that automatically extracts events from the communication channels and adds them to the user's main calendar.
    \item To create a user friendly interface that allows users to automatically add events to the calendar.
    \item To implement smart resolution system that notifies users of scheduling conflicts and provides easy options for resolution.
    \item To integrate all the calendars into Jadwal's single calendar view to make viewing and managing all the events easy.
    \item To prioritize and automatically schedule daily routines such as waking time, sleeping time and prayer time.
    \item To significantly reduce the time users spend on manual calendar management.
\end{itemize}

\section{Scope of the Study}

Jadwal is not just another calendar application; it's a comprehensive time management tool designed to aggregate and optimize your existing calendars and data sources. The scope of the project includes:

\begin{itemize}
    \item Development of an iOS application as the primary platform.
    \item Integration with calendars using CalDAV.
    \item WhatsApp message parsing for event extraction (subject to technical feasibility).
    \item Target audience: Busy professionals, students, and anyone juggling multiple schedules.
    \item User testing phase to ensure ease of use and effectiveness.
\end{itemize}

Our testing methods will include:
\begin{itemize}
    \item Beta testing with a diverse group of users.
    \item Analytics to track user behavior and app performance.
\end{itemize}

\section{Significance of the Study}

Jadwal endeavours to solve problems and its significance can be summarized in the following:

\begin{enumerate}
    \item \textbf{Time is Money}: Time is the only asset you can't get more of, it is being consumed til the last day of your life.
    \item \textbf{Prayer First Calendar}: Prayer times come first, then your daily scheduled items.
    \item \textbf{Streamlined Time Management}: By automatically extracting events from various communication channels, Jadwal significantly reduces the time and effort required for manual calendar management, allowing users to focus on more productive tasks.
    \item \textbf{Reduced Human Error}: Automated event extraction and addition to calendars minimize the risk of missing important events or appointments due to manual input errors or forgetfulness.
    \item \textbf{Integrated Communication and Scheduling}: By bridging the gap between informal communication (e.g., WhatsApp) and formal scheduling, Jadwal addresses a critical pain point in modern time management.
    \item \textbf{Conflict Resolution}: The smart resolution system helps users identify and resolve scheduling conflicts efficiently, reducing stress and improving overall time management.
    \item \textbf{Holistic View of Commitments}: By integrating multiple calendars into a single view, Jadwal provides users with a comprehensive overview of their commitments across various aspects of life, facilitating better decision-making and work-life balance.
\end{enumerate}

\section{Limitations of the Study}

Nothing is perfect, and our project is not a outlier. The limitations we have figured out about it are as follows:

\begin{itemize}
    \item WhatsApp integration allows the app to read the users messages, so it would be hard to prove privacy hasn't been breached.
    \item WhatsApp integration might not always be there, they are a third-party.
    \item Learning new technologies for iOS development might require more time than anticipated.
    \item Accuracy of our algorithms to detect keywords indicating an event agreement has happened, especially for languages other than English.
    \item Time and manpower constraints may limit the number of features we can implement.
    \item Dependency on third-party calendar APIs and their limitations.
\end{itemize}

\section{Organization of the Senior Project}

Our project plan can be illustrated in the following gantt chart, \textbf{Figure \ref{fig:project-gantt-chart}}.

\begin{figure}[!h]
    \centering
    \includegraphics[width=\textwidth]{images/gantt.png}
    \caption{Project Gantt Chart}
    \label{fig:project-gantt-chart}
\end{figure}

\chapter{Literature Review}

In developing Jadwal, we have drawn inspiration from and built upon existing research and products in the field of intelligent calendar management. Some key references include:

\begin{itemize}
    \item \textbf{Clockwise (https://www.getclockwise.com/):} A smart calendar assistant that optimizes schedules and manages team coordination \cite{clockwise}. Clockwise's approach to intelligent time blocking and meeting optimization provides valuable insights for Jadwal's automated scheduling features.
    \item \textbf{Motion (https://www.usemotion.com/):} Motion's Intelligent Calendar takes your meetings, your tasks, your to-do list, your activities, and creates one perfect, optimized schedule to get it all done \cite{motion}.
    \item \textbf{Reclaim AI (https://reclaim.ai/):} An intelligent time management tool that helps optimize schedules and automate tasks \cite{reclaim}.
    \item \textbf{Calendi (https://calendi.ai/):} Calendi describes itself as: ``Calendi is an AI calendar system. Use it for scheduling tasks, automating meetings, and witness the future of calendar.'' \cite{calendi}
    \item \textbf{An Exploratory Study of Calendar Use:} ``Prospective remembering is the use of memory for remembering to do things in the future, as different from retrospective memory functions such as recalling past events.'' \cite{tungare2008exploratorystudycalendaruse}
    \item \textbf{WhatsApp Integration:} Our research indicates that direct WhatsApp integration for event extraction has not been widely implemented in existing calendar applications, making this a unique feature of Jadwal.
\end{itemize}

\begin{figure}[!h]
    \centering
    \includegraphics[width=\textwidth]{images/features-table.png}
    \caption{Feature Comparison Table}
    \label{fig:features-table}
\end{figure}

\chapter{System Analysis and Design}

\section{Functional Requirements}

\begin{itemize}
    \item The user shall be able to access their account using either Google OAuth or magic link via Email. For new users, a new accout is created, and for existing users, they are given access to their account directly
    \item The system shall send a welcome email to new users.
    \item The user should be able to connect a calendar using CalDAV.
    \item The user should be able to connect their WhatsApp account.
    \item The user should be able to add events manually and set priorities optionally.
    \item The user should be able to view integrated calendar.
    \item The user should be able to configure daily routines.
    \item The user should be able to manage scheduling conflicts.
    \item The user should be able to schedule prayer times.
    \item The system shall send event notifications to the user.
    \item The system shall personalize the experience based on answers provided by the users.
    \item The system shall add the WhatsApp extracted events to the calendar. If a conflict occurs, the user shall get a notification to resolve the conflict with suggestions.
    \item The system shall synchronize calendar data across multiple devices.
\end{itemize}

\section{Non-Functional Requirements}

\begin{itemize}
    \item \textbf{Platform Compatibility:} The app shall be compatible with iOS devices running iOS 14.0 or later.
    \item \textbf{Performance:} The app shall load the main calendar view within 3 seconds on 5G with speeds above 200mpbs.
    \item \textbf{User Experience:} The user interface shall follow iOS Human Interface Guidelines for consistency and ease of use.
    \item \textbf{Security:} All data transmissions between the app and servers shall be encrypted using HTTPS.
    \item \textbf{Localization:} The app shall support localization in Arabic and English.
    \item \textbf{Data Privacy:} The app shall comply with the data protection regulations and laws in Saudi Arabia.
\end{itemize}

\section{System use-cases}

\textbf{Figure \ref{fig:use-case-diagram}} shows the use case diagram for the system of Jadwal.

\begin{figure}[!h]
    \centering
    \includegraphics[width=\textwidth]{images/use-case-diagram.png}
    \caption{Use Case Diagram of Jadwal}
    \label{fig:use-case-diagram}
\end{figure}

\begin{usecase}{Continue with Email}
    \ucbasicinfo{\#1}{HIGH}{Regular}
    \ucshortdescription{This UC allows users to login or create an account using their email.}
    \uctrigger{This UC starts when the user enters their email to the system.}
    \ucactors{User}{None}
    \ucpreconditions{User must have an email}
    \ucrelationships{Send Welcome Email}{N/A}{N/A}
    \ucinputsoutputs{
      \begin{itemize}
        \item \textbf{Email} (Source: User)
        \item \textbf{Magic Link (from email)} (Source: User)
      \end{itemize}
    }{
      \begin{itemize}
        \item \textbf{Magic link email} (Destination: User)
        \item \textbf{Confirmation messages} (Destination: User Interface)
      \end{itemize}
    }
    \ucmainflow{
      \begin{enumerate}
        \item The user enters their email.
          \ucinfo{System displays email input field.}
        \item System sends email and displays "Check your email" message.
          \ucinfo{WhatsApp sends the linking code.}
        \item User interacts with email client.
          \ucinfo{Confirmation of successful connection.}
        \item System verifies link and logs user in.
          \ucinfo{System processes link and updates user status.}
      \end{enumerate}
    }
    \ucalternateflows{
      \begin{enumerate}
        \item If the user has no account, the system creates a user.
      \end{enumerate}
    }
    \ucexceptions{
      \begin{itemize}
        \item Invalid email format.
        \item Email not registered (if login only).
        \item Magic link expired or invalid.
      \end{itemize}
    }
    \ucconclusion{This UC ends when the user is logged in.}
    \ucpostconditions{The system generates a JWT.}
    \ucspecialrequirements{An email server must be present to send email magic link.}
\end{usecase}
\begin{usecase}{Continue with Google}
  \ucbasicinfo{HIGH}{Regular}
  \ucshortdescription{This UC allows users to login or sign up with their Google account.}
  \uctrigger{This UC starts when the user clicks "Continue with Google" button in the app.}
  \ucactors{User}{Google}
  \ucpreconditions{The user must have an active Google account.}
  \ucrelationships{Send Welcome Email}{N/A}{N/A}
  \ucinputsoutputs{
    \begin{itemize}
      \item \textbf{Google access token} (Source: User)
    \end{itemize}
  }{
    \begin{itemize}
      \item \textbf{Authentication response} (Destination: User)
    \end{itemize}
  }
  \ucmainflow{
    \begin{enumerate}
      \item The user click continue with Google.
        \ucinfo{App uses OAuth to authenticate with Google}
      \item App sends Google access token to the system.
        \ucinfo{System verifies the token is issued for us and then issues JWT for usage within the app.}
    \end{enumerate}
  }
  \ucalternateflows{
    \begin{itemize}
      \item The user cancels the authentication request.
    \end{itemize}
  }
  \ucexceptions{
    \begin{itemize}
      \item Network issue
    \end{itemize}
  }
  \ucconclusion{This UC ends when the user is logged in.}
  \ucpostconditions{The system generates a JWT.}
\end{usecase}
\begin{usecase}{Connect WhatsApp}
    \ucbasicinfo{\#7}{HIGH}{Regular}
    \ucshortdescription{This UC allows the user to connect their WhatsApp account to our system.}
    \uctrigger{This UC is triggered when the user clicks on "Connect WhatsApp" button in the app.}
    \ucactors{User}{WhatsApp}
    \ucpreconditions{User must be logged in}
    \ucrelationships{N/A}{N/A}{N/A}
    \ucinputsoutputs{
      \begin{itemize}
        \item \textbf{WhatsApp account number} (Source: User)
        \item \textbf{WhatsApp linking code} (Source: User)
      \end{itemize}
    }{
      \begin{itemize}
        \item \textbf{WhatsApp linking code} (Destination: WhatsApp)
        \item \textbf{WhatsApp auth creds} (Destination: System)
      \end{itemize}
    }
    \ucmainflow{
      \begin{enumerate}
        \item The user clicks "Connect WhatsApp" button.
        \ucinfo{System asks for phone number via a dialog}
        \item The user enters their WhatsApp account phone number.
        \ucinfo{WhatsApp sends the linking code.}
        \item The user enters the linking code shown in the WhatsApp app in our app.
        \ucinfo{Confirmation of successful connection}
      \end{enumerate}
    }
    \ucalternateflows{None}
    \ucexceptions{
      \begin{itemize}
        \item \textbf{Wrong linking code:} If the user enters a wrong linking code, the connection of the WhatsApp account will fail unless they enter the correct  code.
        \item \textbf{Network issue:} A network issue interrupting the communication between the app, the server, and WhatsApp.
      \end{itemize}
    }
    \ucpostconditions{The system has access to the user's WhatsApp account.}
    \ucspecialrequirements{None}
    \ucconclusion{The UC ends when the user has a connected WhatsApp account.}
\end{usecase}
\begin{usecase}{Log out}
  \ucbasicinfo{HIGH}{Regular}
  \ucshortdescription{User able to logout}
  \uctrigger{when the UC is done from the application}
  \ucactors{User}{}
  \ucpreconditions{User must log in}
  \ucrelationships{}{}{}
  \ucinputsoutputs{
    \begin{itemize}
      \item \textbf{Log-out} (Source: User)
    \end{itemize}
  }{
    \begin{itemize}
      \item \textbf{logged out from system} (Destination: System)
    \end{itemize}
  }
  \ucmainflow{
    \begin{enumerate}
      \item The user click the button Log-out
        \ucinfo{System displays "Log-out" button}
      \item Emphasize the Log-out.
        \ucinfo{The system may prompt the user to confirm the logout action}
      \item The user will be logged out from the system
        \ucinfo{Confirmation of successful Log-out}
    \end{enumerate}
  }
  \ucalternateflows{
    \begin{itemize}
      \item If the user selects "Cancel" on the confirmation prompt, the system returns to the previous page without logging out.
    \end{itemize}
  }
  \ucexceptions{
    \begin{itemize}
      \item System error
      \item Network issue
    \end{itemize}
  }
  \ucconclusion{}
  \ucpostconditions{The user will be logged out from the system}
  \ucbusinessrules{
    \begin{itemize}
      \item User should log in
    \end{itemize}
  }
  \ucspecialrequirements{Sign-in page should display clear button and error message}
\end{usecase}
\begin{usecase}{Set Event Priority}
    \ucbasicinfo{HIGH}{Regular}
    \ucshortdescription{The user adds or modifies an event, and the system detects a conflict with an existing event in the calendar.}
    \uctrigger{The user adds or modifies an event, and the system detects a conflict with an existing event in the calendar.}
    \ucactors{User}{None}
    \ucpreconditions{The user is logged into the application}
    \ucrelationships{Send Welcome Email}{N/A}{N/A}
    \ucinputsoutputs{
      \begin{itemize}
        \item \textbf{Conflicting events (both the new event and the existing event)} (Source: User)
        \item \textbf{Priority decision} (Source: User)
      \end{itemize}
    }{
      \begin{itemize}
        \item \textbf{The system prioritizes the selected event based on the user's choice
       } (Destination:  Calendar)
        \item \textbf{Confirmation messages} (Destination: User Interface)
      \end{itemize}
    }
    \ucmainflow{
      \begin{enumerate}
        \item Conflict Detection
          \ucinfo{The system detects overlapping events and prompts the user with a conflict}
        \item Choose Priority Option  
          \ucinfo{The system displays a dialogue box with both conflicting events and asks the user to select which event should take priority.}
        \item Resolve Conflict
          \ucinfo{The system offers options to either reschedule the lower-priority event or keep both events with a conflict warning}
        \item Save
          \ucinfo{•	After the user selects the priority, the system saves the decision, either reschedules the lower-priority event or marks it as conflicting and updates the calendar accordingly.}
      \end{enumerate}
    }
    \ucalternateflows{
      \begin{enumerate}
        \item User clicks on a date on the calendar to bring up a popup with event fields and the date is pre-filled. 
        \item The user can add a different color for the events
      \end{enumerate}
    }
    \ucexceptions{
      \begin{itemize}
        \item	No Available Time Slots 
      \end{itemize}
    }
    \ucpostconditions{The conflicting events are resolved, and the calendar reflects the user’s decision on which event to prioritize.}
    \ucspecialrequirements{The user interface must clearly show conflict warnings.}
    \ucconclusion{The use case ends when the user’s priority decision is saved, and the conflicting events are either rescheduled or maintained with a conflict warning.}
\end{usecase}
\begin{usecase}{Schedule Prayer Times}
    \ucbasicinfo{High}{Regular}
    \ucshortdescription{The calendar is blocked and updated automatically according to the person's time zone prayer time.}
    \uctrigger{This usecase triggered when the user enables the prayer time feature in the application.}
    \ucactors{User}{None}
    \ucpreconditions{User must be logged into the system.}
    \ucrelationships{N/A}{N/A}{N/A}
    \ucinputsoutputs{
      \begin{itemize}
        \item \textbf{User's time Zone} (Source: user)
        \item \textbf{We get the IP address of the user and get their time zone}
      \end{itemize}
    }{
      \begin{itemize}
        \item \textbf{The calendar displays the blocked time for prayer time according to their time zone.}
      \end{itemize}
      }
    \ucmainflow{
      \begin{enumerate}
        \item User enables the feature by clicking the the option Schedule Prayer Time.
          \ucinfo{The system checks the time zone of the user and blocks the calendar accordingly.}
      \end{enumerate}
    }
    \ucalternateflows{
      \begin{itemize}
        \item The user doesn't enable the scheduling prayer time.
      \end{itemize}
    }
    \ucexceptions{
      \begin{itemize}
        \item If there's a system error, display a relevant error message.
      \end{itemize}
    }
    \ucpostconditions{The system generates calendar with prayer time}
    \ucspecialrequirements{The system must block the calendar according to their time zone}
    \ucconclusion{User's prayer times are successfully scheduled.}
    \ucbusinessrules{
      \begin{itemize}
        \item Prayer times must be within valid time ranges.
      \end{itemize}
    }
\end{usecase}
\begin{usecase}{Receive Event Notifications}
  \ucbasicinfo{High}{Regular}
  \ucshortdescription{Users receive notifications about upcoming events.}
  \uctrigger{The UC is triggered when the choosen time of an event's reminder has arrived}
  \ucactors{User}{None}
  \ucpreconditions{User must be logged into the system and set a reminder of the specific event.}
  \ucrelationships{N/A}{N/A}{N/A}
  \ucinputsoutputs{
    \begin{itemize}
      \item \textbf{Time of the event reminder} (Source: User)
    \end{itemize}
  }{
    \begin{itemize}

      \item \textbf{Notifications sent to users.} (Destination: System)
    \end{itemize}
  }
  \ucmainflow{
    \begin{enumerate}
      \item The system checks the alarms set for every event.
            \ucinfo{The system checks every 1 minute for set alarms for every event.}
      \item Notifications sent to users.
            \ucinfo{The user is reminded about the event by sending the notification}
    \end{enumerate}
  }
  \ucalternateflows{
    \begin{itemize}
      \item \textbf{If user denys the access to the notification the notifications are not sent.}
    \end{itemize}
  }
  \ucexceptions{
    \begin{itemize}
      \item \textbf{Network issue}
    \end{itemize}
  }
  \ucconclusion{The system checks for set alarm every 1 minute and if the event is detected the system sends the notification.}
  \ucpostconditions{Notifications are sent.}
  \ucspecialrequirements{Notification permission.}
  \ucbusinessrules{The system has to check every minute for the set alarm.
  }
\end{usecase}

\begin{figure}[!h]
  \centering
  \includegraphics[width=\textwidth]{images/docs/diagrams/sequence-diagrams/all-sequence-diagrams/Receive Event Notifications.png}
  \caption{Receive Event Notifications Sequence Diagram}
  \label{fig:seq/receive-event-notifications}
\end{figure}

The ``Receive Event Notifications Sequence Diagram'', shown in \textbf{Figure~\ref{fig:seq/receive-event-notifications}}, illustrates Jadwal's continuous event notification monitoring and delivery system. The sequence operates in a continuous polling loop that executes every minute, ensuring timely notification delivery for all scheduled events.

The polling process consists of several key steps:
\begin{enumerate}
  \item The System queries the Database for active alarms through regular checks
  \item For each batch of active alarms found:
        \begin{itemize}
          \item Retrieves associated device IDs for notification targets
          \item Iterates through each event requiring notification
          \item Dispatches push notifications via Apple Push Notification service (APNs)
        \end{itemize}
  \item If no active alarms are found, the system continues its polling cycle
\end{enumerate}

This robust notification system ensures reliable delivery of event reminders while efficiently managing system resources. The one-minute polling interval provides a balance between timely notifications and system performance, while the batch processing of notifications optimizes the interaction with APNs. The system's ability to handle multiple device IDs per user ensures notifications reach users across all their registered devices.

% Bibliography
\bibliography{references}
\bibliographystyle{apalike}

\end{document}