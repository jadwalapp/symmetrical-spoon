\begin{usecase}{Suggest Conflict Resolutions}
  \ucbasicinfo{High}{Regular}
  \ucshortdescription{This UC gives the user all the conflicts and possible ways to resolve it.}
  \uctrigger{The UC is triggered when a conflict is detected between any overlapping event}
  \ucactors{User}{WhatsApp}
  \ucpreconditions{The calendar must have events}
  \ucrelationships{N/A}{Manage Scheduling Conflicts}{N/A}
  \ucinputsoutputs{
    \begin{itemize}
      \item Overlapping events (Source: User or WhatsApp extraction)
      \item User suggested resolution (Source: User)
    \end{itemize}
  }{
    \begin{itemize}
      \item {Resolution options} (Destination: User Interface)
      \item {Updated calendar schedule} (Destination: Calendar)
    \end{itemize}
  }
  \ucmainflow{
    \begin{enumerate}
      \item The system detection conflicts 
        \ucinfo{The system detects overlapping of events either added manually or extracted from WhatsApp and gives a notification to the user.}
      \item The system gives the suggestions for Conflicts.
        \ucinfo{The system provides the user with the list of resolution options.
        \begin{itemize}
          \item By moving the overlapping event to another time slot.   
          \item Keep both events with a conflict warning. 
        \end{itemize}}
    \end{enumerate}
  }
  \ucalternateflows{
    \begin{enumerate}
      \item {No conflicts detected}
    \end{enumerate}
  }
  \ucexceptions{
    \begin{itemize}
      \item If the system doesn't get any possible way to resolve the conflict then the system would mark both the event as conflicting.
    \end{itemize}
  }
  \ucconclusion{The UC ends when the user chooses resolution weather if to reschedule the event or leaving it without resolving and it is reflected in on the calendar.}
  \ucpostconditions{The conflicting events in the calendar are either resolved or marked as conflicting}
  \ucspecialrequirements{The system give feasible conflict resolution options}
\end{usecase}
