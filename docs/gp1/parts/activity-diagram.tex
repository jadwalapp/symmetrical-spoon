\section{Activity Diagram}

Activity diagrams provide a visual representation of the workflow and business processes within a system. For Jadwal, the activity diagram maps out the user's journey through the application, from authentication to daily interactions, illustrating the various paths and decision points users encounter. This diagram is particularly important as it demonstrates how different features of Jadwal—such as calendar management through EventKit and local notifications—interconnect in actual usage.

An activity diagram is shown in Figure~\ref{fig:activity-diagram} above, and it illustrates the flows the user of the app can take. It starts with the authentication which supports both Google and Email.

Then after the token is verified, the app requests calendar access permissions through EventKit. Once granted, the user can view their calendars and manage events. When viewing the calendar, they can view an event or add one. Adding an event might result in a conflict, so if one arises, they can try editing it and adding it again until no conflicts arise.

After any of the previous two steps, the user can go back to view calendar or open settings page. If the choose to open settings page, they have three actions available. They can connect a CalDAV account, connect WhatsApp, or logout. Starting with the connecting a WhatsApp account, the system will extract events from the user's WhatsApp account automatically. If any conflicts arise, the system suggests a conflict resolution, and the user can take action by managing the scheduling conflict. But if no conflicts arise, the system just continues normally.

Another option for the user is connecting a CalDAV account, and once that happens, the user can go back to having the option of going to calendar view or open settings page.

Finally, the logout action and that terminates the session and the user is logged out of the app.

\begin{figure}[!h]
    \centering
    \includegraphics[width=0.9\textwidth]{images/activity-diagram.png}
    \caption{Activity Diagram of Jadwal}
    \label{fig:activity-diagram}
\end{figure}