\chapter{System Analysis and Design}

The development of a complex calendar management system like Jadwal requires careful analysis of requirements and thoughtful system design to ensure robust functionality and seamless user experience. This chapter presents a detailed examination of Jadwal's architecture, from its core requirements to the intricate relationships between system components. Through use case diagrams, activity diagrams, class diagrams, and database design (both ER and Relational), we provide a comprehensive blueprint of how Jadwal transforms its innovative concepts into a practical, functioning system.

\section{Introduction}

A well-designed system requires a good understanding of both functional and non-functional requirements to meet user expectations and deliver a seamless experience.

This section shows the functional and non-functional requirements which is the backbone of Jadwal's development. The functional requirements focus on core features, such as user authentication, calendar integration and event management. Ensuring the users can effectively manage their schedules. The non-functional requirements focuses on performance, security, compatibility, and user experience, ensuring the application stands well with the industry standards by providing efficient interface.

Combining all these requirements helps in the design and implementation of Jadwal which will lead to a better application solving real issues and meeting the needs of the user.

\section{Functional Requirements}

The following requirements outline the core features and capabilities that Jadwal must provide to fulfill its purpose as an intelligent calendar management system:
\begin{itemize}
    \item The user shall be able to access their account using either Google OAuth or magic link via Email. For new users, a new account is created, and for existing users, they are given access to their account directly.
    \item The system shall send a welcome email to new users.
    \item The user should be able to create a calendar.
    \item The user should be able to connect a calendar using CalDAV.
    \item The user should be able to connect their WhatsApp account.
    \item The user should be able to add events manually.
    \item The user should be able to view integrated calendar.
    \item The user should be able to manage scheduling conflicts.
    \item The user should be able to schedule prayer times.
    \item The system shall send event notifications to the user.
    \item The system shall add the WhatsApp extracted events to the calendar. If a conflict occurs, the user shall get a notification to resolve the conflict with suggestions.
\end{itemize}

Each functional requirement listed above will be explained in details through a use case description in the coming sections.

\newpage

\section{Non-Functional Requirements}

While functional requirements define what the system does, non-functional requirements specify how the system performs its functions. These requirements focus on the quality attributes, performance standards, and technical constraints that ensure Jadwal delivers a reliable, secure, and user-friendly experience.

\begin{itemize}
    \item \textbf{Platform Compatibility:} The app shall be compatible with iOS devices running iOS 16.0 or later.
    \item \textbf{Performance:} The app shall load the main calendar view within 3 seconds on 5G with speeds above 200mpbs.
    \item \textbf{User Experience:} The user interface shall follow iOS Human Interface Guidelines for consistency and ease of use.
    \item \textbf{Security:} All data transmissions between the app and servers shall be encrypted using HTTPS.
\end{itemize}


\section{Security Architecture}

In today's digital world, security is a key concern for any application that handles user data. Jadwal places a strong emphasis on ensuring the security and trustworthiness of the platform for its users, implementing multiple layers of security measures to protect user data.

\subsection{Authentication: Magic Link and JWT Tokens}

To provide secure authentication, Jadwal implements Magic Link authentication. Instead of relying on traditional username and password combinations, which is vulnerable to various attacks, such as credentials leakage through database dumps. To mitigate this, Jadwal sends the users trying to authenticate a secure magic link to their verified email address. A magic link is a special URL that contains a secure token. For example:

\begin{verbatim}
    https://jadwal.app/magic-link?token=some-uuid
\end{verbatim}

In this URL, \texttt{some-uuid} is the secure token, and the complete URL is what we call the magic link. When sent to the user's email, clicking this link proves they have access to the email account they're trying to use.

When a user logs in using the Magic Link sent to their email, Jadwal secures the user's account by verifying the magic token provided. This ensures that the account can only be accessed by the legitimate user who has access to the registered email account. To enhance security, the magic token has a limited lifetime of 15 minutes, significantly reducing the window of opportunity for potential attacks.

Upon successful magic link verification, Jadwal issues a JWT (JSON Web Token) signed with the platform's private key. This digital signature serves as a cryptographic guarantee of the token's authenticity, allowing the system to verify that tokens haven't been tampered with and were legitimately issued by Jadwal.

\subsection{Token Storage and Security}

To ensure the security of authentication tokens, Jadwal implements secure storage practices for storing the magic token in the database. Magic tokens are never stored in their original form; instead, they are protected using the SHA-256 hashing algorithm before being saved in the database. When verifying user tokens, the system hashes the user-provided token and compares it with the stored hash, ensuring that even in the unlikely event of a database breach, the original tokens remain secure.

\subsection{Secure Logout Implementation}

Jadwal implements a secure logout mechanism by removing the authentication token from the user's device upon logout. This practice ensures that once a user logs out, their session token cannot be reused for unauthorized access, maintaining the integrity of user sessions.

\subsection{Transparency Through Open Source}

As part of our commitment to security and privacy, Jadwal will be released as open-source software. This transparency allows security experts and users to verify our security implementations and privacy practices. Users can inspect exactly how their data is handled and verify that our privacy commitments are upheld through code review.


\section{System Architecture}

Jadwal implements a modern, distributed architecture leveraging gRPC (Google Remote Procedure Call) for efficient communication between its components. The system is divided into several key services, each responsible for specific functionality while maintaining high performance and reliability.

\begin{figure}[!h]
    \centering
    \includegraphics[width=0.6\textwidth]{images/architecture.png}
    \caption{Jadwal System Architecture}
    \label{fig:architecture}
\end{figure}

The architecture, shown in \textbf{Figure~\ref{fig:architecture}}, consists of the following main components:

\begin{enumerate}
    \item \textbf{Frontend (Mishkat)}
          \begin{itemize}
              \item iOS SwiftUI application implementing the client-side gRPC communication
              \item Handles user interface and local state management
          \end{itemize}

    \item \textbf{Backend (Falak)}
          \begin{itemize}
              \item Core gRPC server implementing the primary business logic
              \item Auth Service managing user authentication and session management
              \item Calendar Service handling calendar operations and integration
              \item CalDAV Client enabling connection to external calendar services
              \item Event Consumer processing WhatsApp events from the message queue
          \end{itemize}

    \item \textbf{WhatsApp Service}
          \begin{itemize}
              \item Node.js gRPC server managing WhatsApp communication
              \item WhatsApp Web.js client for message monitoring
              \item Event Producer publishing detected events to the message queue
          \end{itemize}

    \item \textbf{Message Queue}
          \begin{itemize}
              \item RabbitMQ handling asynchronous event processing
              \item Ensures reliable delivery of WhatsApp events to the backend
          \end{itemize}

    \item \textbf{Database}
          \begin{itemize}
              \item PostgreSQL storing user data, calendar information, and events
              \item Maintains data consistency across all services
          \end{itemize}
\end{enumerate}

This architecture enables several key benefits:

\begin{itemize}
    \item \textbf{Performance}: gRPC's use of Protocol Buffers and HTTP/2 ensures efficient communication between services
    \item \textbf{Scalability}: Separate services can be scaled independently based on load
    \item \textbf{Reliability}: Message queue ensures no events are lost during processing
    \item \textbf{Maintainability}: Clear separation of concerns makes the system easier to maintain and update
\end{itemize}

\section{System Use Cases}
The functionality of Jadwal can be best understood through its various use cases, which demonstrate how users interact with the system. Each use case details specific interactions and flows that make up the core functionality of Jadwal. The diagram in Figure~\ref{fig:use-case-diagram} provides an overview of all use cases and their relationships.

\textbf{Figure~\ref{fig:use-case-diagram}} shows the complete use case diagram for Jadwal's system, illustrating the relationships between these fourteen distinct use cases and their actors.

\begin{figure}[!h]
    \centering
    \includegraphics[width=\textwidth]{images/use-case-diagram.png}
    \caption{Use Case Diagram of Jadwal}
    \label{fig:use-case-diagram}
\end{figure}

\subsection{Authentication and User Management}
User authentication is the first interaction point with Jadwal. We support both email-based authentication through magic links and Google OAuth to provide secure and convenient access options. The following use cases detail the login flows and account management features.

\begin{usecase}{Continue with Email}
    \ucbasicinfo{\#1}{HIGH}{Regular}
    \ucshortdescription{This UC allows users to login or create an account using their email.}
    \uctrigger{This UC starts when the user enters their email to the system.}
    \ucactors{User}{None}
    \ucpreconditions{User must have an email}
    \ucrelationships{Send Welcome Email}{N/A}{N/A}
    \ucinputsoutputs{
      \begin{itemize}
        \item \textbf{Email} (Source: User)
        \item \textbf{Magic Link (from email)} (Source: User)
      \end{itemize}
    }{
      \begin{itemize}
        \item \textbf{Magic link email} (Destination: User)
        \item \textbf{Confirmation messages} (Destination: User Interface)
      \end{itemize}
    }
    \ucmainflow{
      \begin{enumerate}
        \item The user enters their email.
          \ucinfo{System displays email input field.}
        \item System sends email and displays "Check your email" message.
          \ucinfo{WhatsApp sends the linking code.}
        \item User interacts with email client.
          \ucinfo{Confirmation of successful connection.}
        \item System verifies link and logs user in.
          \ucinfo{System processes link and updates user status.}
      \end{enumerate}
    }
    \ucalternateflows{
      \begin{enumerate}
        \item If the user has no account, the system creates a user.
      \end{enumerate}
    }
    \ucexceptions{
      \begin{itemize}
        \item Invalid email format.
        \item Email not registered (if login only).
        \item Magic link expired or invalid.
      \end{itemize}
    }
    \ucconclusion{This UC ends when the user is logged in.}
    \ucpostconditions{The system generates a JWT.}
    \ucspecialrequirements{An email server must be present to send email magic link.}
\end{usecase}
\begin{usecase}{Continue with Google}
  \ucbasicinfo{HIGH}{Regular}
  \ucshortdescription{This UC allows users to login or sign up with their Google account.}
  \uctrigger{This UC starts when the user clicks "Continue with Google" button in the app.}
  \ucactors{User}{Google}
  \ucpreconditions{The user must have an active Google account.}
  \ucrelationships{Send Welcome Email}{N/A}{N/A}
  \ucinputsoutputs{
    \begin{itemize}
      \item \textbf{Google access token} (Source: User)
    \end{itemize}
  }{
    \begin{itemize}
      \item \textbf{Authentication response} (Destination: User)
    \end{itemize}
  }
  \ucmainflow{
    \begin{enumerate}
      \item The user click continue with Google.
        \ucinfo{App uses OAuth to authenticate with Google}
      \item App sends Google access token to the system.
        \ucinfo{System verifies the token is issued for us and then issues JWT for usage within the app.}
    \end{enumerate}
  }
  \ucalternateflows{
    \begin{itemize}
      \item The user cancels the authentication request.
    \end{itemize}
  }
  \ucexceptions{
    \begin{itemize}
      \item Network issue
    \end{itemize}
  }
  \ucconclusion{This UC ends when the user is logged in.}
  \ucpostconditions{The system generates a JWT.}
\end{usecase}
\begin{usecase}{Send Welcome Email}
  \ucbasicinfo{Low}{Regular}
  \ucshortdescription{This UC welcomes the user to the platform.}
  \uctrigger{This UC starts when the user account is created.}
  \ucactors{User}{None}
  \ucpreconditions{User account must be created in the system.}
  \ucrelationships{N/A}{N/A}{N/A}
  \ucinputsoutputs{
    \begin{itemize}
      \item \textbf{User name} (Source: System)
      \item \textbf{Welcome email template} (Source: System)
    \end{itemize}
  }{
    \begin{itemize}
      \item \textbf{Welcome email} (Destination: User)
    \end{itemize}
  }
  \ucmainflow{
    \begin{enumerate}
      \item The system fetches the user information.
            \ucinfo{The database is used.}
      \item The system fetches the send welcome email template.
            \ucinfo{The template is filled with the user name.}
      \item The system sends the email with the template.
            \ucinfo{The email is received by the user welcoming them.}
    \end{enumerate}
  }
  \ucexceptions{
    \begin{itemize}
      \item Email server is down.
    \end{itemize}
  }
  \ucconclusion{This UC ends when the user receives an email from us welcoming them.}
  \ucspecialrequirements{An email server must be present to send welcome email.}
\end{usecase}

\begin{figure}[!h]
  \centering
  \includegraphics[width=\textwidth]{images/docs/diagrams/sequence-diagrams/all-sequence-diagrams/Send Welcome Email.png}
  \caption{Send Welcome Email Sequence Diagram}
  \label{fig:seq/send-welcome-email}
\end{figure}

The "sending a welcome email Sequence Diagram", shown in \textbf{Figure~\ref{fig:seq/send-welcome-email}}, shows the process of sending a welcome email to new customers. It begins with the system retrieving user data from the database using the FetchUserInfo function. Once the user data is fetched, the system requests a magic link email template via Get Magic Link Template. The template is then customized with the retrieved user data using Fill Magic Link Template by UserData. The system sends the filled email template to the email server via SendEmail, which delivers the email. Finally, the email server responds with EmailSentResponse, confirming the email's successful dispatch. This sequence ensures personalized and reliable email delivery.

\clearpage
\begin{usecase}{Log out}
  \ucbasicinfo{HIGH}{Regular}
  \ucshortdescription{User able to logout}
  \uctrigger{when the UC is done from the application}
  \ucactors{User}{}
  \ucpreconditions{User must log in}
  \ucrelationships{}{}{}
  \ucinputsoutputs{
    \begin{itemize}
      \item \textbf{Log-out} (Source: User)
    \end{itemize}
  }{
    \begin{itemize}
      \item \textbf{logged out from system} (Destination: System)
    \end{itemize}
  }
  \ucmainflow{
    \begin{enumerate}
      \item The user click the button Log-out
        \ucinfo{System displays "Log-out" button}
      \item Emphasize the Log-out.
        \ucinfo{The system may prompt the user to confirm the logout action}
      \item The user will be logged out from the system
        \ucinfo{Confirmation of successful Log-out}
    \end{enumerate}
  }
  \ucalternateflows{
    \begin{itemize}
      \item If the user selects "Cancel" on the confirmation prompt, the system returns to the previous page without logging out.
    \end{itemize}
  }
  \ucexceptions{
    \begin{itemize}
      \item System error
      \item Network issue
    \end{itemize}
  }
  \ucconclusion{}
  \ucpostconditions{The user will be logged out from the system}
  \ucbusinessrules{
    \begin{itemize}
      \item User should log in
    \end{itemize}
  }
  \ucspecialrequirements{Sign-in page should display clear button and error message}
\end{usecase}

\subsection{Calendar Management}
At its core, Jadwal helps users manage their calendars effectively. These use cases show how users can create new calendars, connect existing ones through CalDAV, and view all their calendars in one integrated interface.

\begin{usecase}{Connect Calendar}
  \ucbasicinfo{Medium}{Regular}
  \ucshortdescription{This UC allows the user to add Jadwal's Baikal calendar credentials using a .mobileconfig profile to their iOS device calendar accounts.}
  \uctrigger{This UC is triggered when the user wants clicks ``Easy Setup'' in the mobile app.}
  \ucactors{User}{iOS Settings}
  \ucpreconditions{User must be logged in}
  \ucrelationships{N/A}{N/A}{N/A}
  \ucinputsoutputs{
    \begin{itemize}
      \item \textbf{Magic Token with type CalDav} (Source: System)
      \item \textbf{.mobileconfig profile} (Source: System)
    \end{itemize}
  }{
    \begin{itemize}
      \item \textbf{Calendar configuration status} (Destination: iOS Settings)
    \end{itemize}
  }
  \ucmainflow{
    \begin{enumerate}
      \item The user taps "Easy Setup" in the app.
            \ucinfo{The app calls the backend to get a Magic Token of type CalDAV.}
      \item The app prompts the user to download the file from the backend endpoint.
            \ucinfo{The Magic Token is used to authenticate the user and get his credentials and give him his unique file to download.}
      \item The user approves the profile installation.
            \ucinfo{iOS configures the CalDAV account with our Baikal server automatically.}
      \item The app shows the user a success page when he goes back to it.
            \ucinfo{Success is shown only after confirming the calendar configuration is complete.}
    \end{enumerate}
  }
  \ucalternateflows{
    \begin{enumerate}
      \item If the user denies profile installation:
            \begin{itemize}
              \item The app shows an error page with "Setup cancelled - Try again later"
              \item The user can retry the setup process later
            \end{itemize}
      \item If the calendar configuration fails:
            \begin{itemize}
              \item The app shows an error page with "Calendar setup failed"
              \item The user may need to contact support or retry the process
            \end{itemize}
    \end{enumerate}
  }

  \ucconclusion{The UC ends when either the calendar is successfully configured or an error page is shown to the user.}
  \ucpostconditions{Either the user's iOS device is configured to sync with our Baikal calendar server, or the user is informed of the failure with appropriate guidance.}
  \ucspecialrequirements{The system must generate secure, user-specific .mobileconfig profiles.}
\end{usecase}

\begin{figure}[!h]
  \centering
  \includegraphics[width=\textwidth]{images/docs/diagrams/sequence-diagrams/all-sequence-diagrams/Connect Calendar.png}
  \caption{Connect Calendar Sequence Diagram}
  \label{fig:seq/connect-calendar}
\end{figure}

The ``Connect Calendar Sequence Diagram'', shown in \textbf{Figure~\ref{fig:seq/connect-calendar}}, illustrates the streamlined process of connecting an iOS device to Jadwal's Baikal calendar server. The sequence begins when the user taps ``Easy Setup'' in the app.

The process follows a secure flow where the app first requests a Magic Token of type CalDAV from the backend. Using this token, the app then requests a user-specific \textit{.mobileconfig} profile that contains the pre-configured CalDAV credentials. When the user downloads this profile, iOS Settings takes over to handle the secure installation process.

The diagram illustrates the following possible paths:
\begin{enumerate}
  \item \textbf{Success Path:} The user approves the profile installation in iOS Settings, which then automatically configures the CalDAV account. Upon returning to the app, the user is shown a success page after the system confirms the calendar configuration is complete.
  \item \textbf{User Denial Path:} If the user denies the profile installation, they return to the app which displays "Setup cancelled - Try again later", allowing them to retry the process at their convenience.
  \item \textbf{Configuration Failure Path:} If the profile installation is approved but the calendar configuration fails, the app shows a "Calendar setup failed" message, prompting the user to either contact support or retry the process.
\end{enumerate}

This implementation leverages iOS's native configuration profile system to ensure a secure and user-friendly setup process. The use of Magic Tokens and encrypted \textit{.mobileconfig} profiles during transit guarantees that calendar credentials are transmitted and stored securely on the user's device. The clear error handling paths ensure users receive appropriate guidance when issues occur during any stage of the setup process.
\begin{usecase}{Create Calendar}
  \ucbasicinfo{High}{Regular}
  \ucshortdescription{This UC allows the user to create a new calendar using iOS EventKit.}
  \uctrigger{This UC is triggered when the user clicks ``Create Calendar'' in the app.}
  \ucactors{User}{EventKit}
  \ucpreconditions{
    \begin{itemize}
      \item User must be logged in
      \item Calendar access must be authorized in iOS Settings
      \item Baikal CalDAV account must be configured
    \end{itemize}
  }
  \ucrelationships{N/A}{N/A}{N/A}
  \ucinputsoutputs{
    \begin{itemize}
      \item \textbf{Calendar name} (Source: User)
      \item \textbf{Calendar account to add calendar under} (Source: User)
      \item \textbf{Calendar color} (Source: User)
    \end{itemize}
  }{
    \begin{itemize}
      \item \textbf{New calendar} (Destination: EventKit)
      \item \textbf{Creation status} (Destination: App)
    \end{itemize}
  }
  \ucmainflow{
    \begin{enumerate}
      \item The user taps the ``Calendar'' icon button in the app toolbar in the ``Calendar'' screen.
            \ucinfo{The app presents a ``Calendars'' sheet.}
      \item The user clicks the ``plus'' icon button in the toolbar of the ``Calendars'' sheet.
            \ucinfo{The app presents a form for calendar name, selector to choose account to add calendar under, and calendar color.}
      \item The user enters calendar name, selects a color, and selector to choose account to add calender under.
            \ucinfo{The app uses EventKit to create a new calendar.}
      \item EventKit creates the calendar and handles synchronization.
            \ucinfo{The calendar appears in the user's calendar list.}
    \end{enumerate}
  }
  \ucalternateflows{
    \begin{itemize}
      \item If calendar creation fails, the app shows an error and allows retry.
    \end{itemize}
  }
  \ucexceptions{
    \begin{itemize}
      \item \textbf{EventKit access denied:} The app prompts to enable calendar access in Settings.
      \item \textbf{Network issues:} EventKit handles synchronization internally.
    \end{itemize}
  }
  \ucconclusion{The UC ends when EventKit confirms the calendar creation.}
  \ucpostconditions{A new calendar is created through EventKit.}
  \ucspecialrequirements{The system must use EventKit for all calendar operations.}
\end{usecase}

\begin{figure}[!h]
  \centering
  \includegraphics[width=0.5\textwidth]{images/docs/diagrams/sequence-diagrams/all-sequence-diagrams/Create Calendar.png}
  \caption{Create Calendar Sequence Diagram}
  \label{fig:seq/create-calendar}
\end{figure}

The ``Create Calendar Sequence Diagram'', shown in \textbf{Figure~\ref{fig:seq/create-calendar}}, illustrates the process of creating a new calendar in the Jadwal app. The sequence begins when the user accesses the calendar creation interface through the ``Calendars'' sheet, accessed via the calendar icon in the toolbar.

The flow involves several key steps:
\begin{enumerate}
  \item The user provides essential calendar details through a form interface:
        \begin{itemize}
          \item Calendar name
          \item Account selection (where to add the calendar)
          \item Calendar color
        \end{itemize}
  \item The app communicates with EventKit to create the new calendar
  \item EventKit handles the calendar creation and all necessary synchronization
\end{enumerate}

The process is designed to be robust and user-friendly:
\begin{itemize}
  \item If EventKit access is denied, the app guides users to enable calendar access in iOS Settings
  \item If creation fails, the app shows an error and allows retry
  \item EventKit handles all synchronization internally
\end{itemize}

This implementation leverages iOS's native EventKit framework, which manages all calendar operations and synchronization internally. This provides a reliable and native iOS experience while ensuring proper calendar management.
\begin{usecase}{View Integrated Calendar}
    \ucbasicinfo{High}{Regular}
    \ucshortdescription{This UC allows the user to view their calendar events from all configured calendars through iOS's native calendar system.}
    \uctrigger{This UC is triggered when the user opens the calendar view in the app.}
    \ucactors{User}{EventKit}
    \ucpreconditions{
        \begin{itemize}
            \item User must be logged in
            \item User must have granted calendar access permission
            \item User must have at least one calendar configured in iOS
        \end{itemize}
    }
    \ucrelationships{N/A}{N/A}{N/A}
    \ucinputsoutputs{
        \begin{itemize}
            \item \textbf{Calendar view preferences} (Source: User)
            \item \textbf{Calendar events} (Source: iOS Calendar System)
        \end{itemize}
    }{
        \begin{itemize}
            \item \textbf{Integrated calendar view} (Destination: App)
        \end{itemize}
    }
    \ucmainflow{
        \begin{enumerate}
            \item The user opens the calendar view in the app.
                  \ucinfo{The app uses EventKit to fetch events from all configured calendars.}
            \item The app displays the integrated calendar view.
                  \ucinfo{Events from all calendars are shown in a unified view.}
            \item The user can interact with the calendar view.
                  \ucinfo{The user can view event details, switch between different view modes (day, week, month), and filter calendars.}
        \end{enumerate}
    }
    \ucalternateflows{
        \begin{itemize}
            \item If no calendars are configured, the app shows an empty state with instructions to add calendars through iOS settings.
            \item If calendar access is revoked, the app prompts the user to grant access in iOS settings.
        \end{itemize}
    }
    \ucexceptions{
        \begin{itemize}
            \item \textbf{Calendar access denied:} If the app doesn't have calendar access permission.
            \item \textbf{Event fetch failure:} If EventKit fails to retrieve calendar events.
        \end{itemize}
    }
    \ucconclusion{The UC ends when the user has a clear view of their calendar events from all configured calendars.}
    \ucpostconditions{The user can view and interact with events from all their configured calendars in a unified interface.}
\end{usecase}

\begin{figure}[!h]
    \centering
    \includegraphics[width=0.5\textwidth]{images/docs/diagrams/sequence-diagrams/all-sequence-diagrams/View Integrated Calendar.png}
    \caption{View Integrated Calendar Sequence Diagram}
    \label{fig:seq/view-integrated-calendar}
\end{figure}

The ``View Integrated Calendar Sequence Diagram'', shown in \textbf{Figure~\ref{fig:seq/view-integrated-calendar}}, illustrates how the app uses EventKit to fetch and display calendar events from iOS's native calendar system.

The process involves:
\begin{enumerate}
    \item The app requests calendar events through EventKit when the user opens the calendar view
    \item EventKit fetches events from all configured calendars in iOS
    \item The app displays the events in an integrated view
    \item The user can interact with the events and switch between different view modes
\end{enumerate}

This approach leverages iOS's built-in calendar infrastructure to provide a seamless calendar viewing experience.

\subsection{WhatsApp Integration and Event Extraction}
One of Jadwal's key features is its ability to automatically extract events from WhatsApp conversations. These use cases explain how users connect their WhatsApp account and how the system processes messages to identify and add events to their calendar.

\begin{usecase}{Connect WhatsApp}
    \ucbasicinfo{\#7}{HIGH}{Regular}
    \ucshortdescription{This UC allows the user to connect their WhatsApp account to our system.}
    \uctrigger{This UC is triggered when the user clicks on "Connect WhatsApp" button in the app.}
    \ucactors{User}{WhatsApp}
    \ucpreconditions{User must be logged in}
    \ucrelationships{N/A}{N/A}{N/A}
    \ucinputsoutputs{
      \begin{itemize}
        \item \textbf{WhatsApp account number} (Source: User)
        \item \textbf{WhatsApp linking code} (Source: User)
      \end{itemize}
    }{
      \begin{itemize}
        \item \textbf{WhatsApp linking code} (Destination: WhatsApp)
        \item \textbf{WhatsApp auth creds} (Destination: System)
      \end{itemize}
    }
    \ucmainflow{
      \begin{enumerate}
        \item The user clicks "Connect WhatsApp" button.
        \ucinfo{System asks for phone number via a dialog}
        \item The user enters their WhatsApp account phone number.
        \ucinfo{WhatsApp sends the linking code.}
        \item The user enters the linking code shown in the WhatsApp app in our app.
        \ucinfo{Confirmation of successful connection}
      \end{enumerate}
    }
    \ucalternateflows{None}
    \ucexceptions{
      \begin{itemize}
        \item \textbf{Wrong linking code:} If the user enters a wrong linking code, the connection of the WhatsApp account will fail unless they enter the correct  code.
        \item \textbf{Network issue:} A network issue interrupting the communication between the app, the server, and WhatsApp.
      \end{itemize}
    }
    \ucpostconditions{The system has access to the user's WhatsApp account.}
    \ucspecialrequirements{None}
    \ucconclusion{The UC ends when the user has a connected WhatsApp account.}
\end{usecase}
\clearpage
\begin{usecase}{Extract Events from WhatsApp}
  \ucbasicinfo{High}{Regular}
  \ucshortdescription{Allows users to extract event details shared in WhatsApp messages and save them to their calendar.}
  \uctrigger{User selects a message containing event details in WhatsApp.}
  \ucactors{User}{WhatsApp}
  \ucpreconditions{User must have WhatsApp installed and linked to the system.}
  \ucrelationships{N/A}{Event Management}{N/A}
  \ucinputsoutputs{
    \begin{itemize}
      \item \textbf{User-selected WhatsApp message containing event details.} (Source: User)
    \end{itemize}
  }{
    \begin{itemize}
      \item \textbf{Extracted event details saved to the calendar.} (Destination: User)
    \end{itemize}
  }
  \ucmainflow{
    \begin{enumerate}
      \item User opens WhatsApp and selects a message with event details. 
        \ucinfo{Display of the selected message.}
      \item User taps on the ``Extract Event'' option.
        \ucinfo{System processes the message to identify event details.}
      \item System extracts event details from the message.
        \ucinfo{Date, time, and location are parsed from the text. }
    \end{enumerate}
  }
  \ucalternateflows{
    \begin{itemize}
      \item If the message does not contain valid event details, display an error message.
      \item Notify the user about invalid format. 
    \end{itemize}
  }
  \ucexceptions{
    \begin{itemize}
      \item If there is a system error during extraction, display a relevant error message. 
      \item Log error for troubleshooting.
    \end{itemize}
  }
  \ucconclusion{User successfully extracts event details from WhatsApp and saves them to the calendar.}
  \ucpostconditions{The event is added to the user's calendar, and the user can view it in their schedule.}
  \ucbusinessrules{
    \begin{itemize}
      \item Only messages with recognized event formats can be extracted.
    \end{itemize}
  }
  \ucspecialrequirements{The system must have permissions to access WhatsApp messages and extract relevant information.}
\end{usecase}

\subsection{Event Management and Conflict Resolution}
Managing events and resolving scheduling conflicts are daily challenges for users. These use cases demonstrate how Jadwal handles event creation, conflict detection, and provides smart resolution options.

\begin{usecase}{Suggest Conflict Resolutions}
  \ucbasicinfo{High}{Regular}
  \ucshortdescription{This UC gives the user all the conflicts and possible ways to resolve it.}
  \uctrigger{The UC is triggered when a conflict is detected between any overlapping event}
  \ucactors{User}{WhatsApp}
  \ucpreconditions{The calendar must have events}
  \ucrelationships{N/A}{Manage Scheduling Conflicts}{N/A}
  \ucinputsoutputs{
    \begin{itemize}
      \item \textbf{Overlapping events} (Source: Saved Events in Database)
      \item \textbf{User suggested resolution} (Source: User)
    \end{itemize}
  }{
    \begin{itemize}
      \item \textbf{Resolution options} (Destination: User Interface)
      \item \textbf{Updated calendar schedule} (Destination: Calendar)
    \end{itemize}
  }
  \ucmainflow{
    \begin{enumerate}
      \item The system detection conflicts
            \ucinfo{The system detects overlapping of events either added manually or extracted from WhatsApp and gives a notification to the user.}
      \item The system gives the suggestions for Conflicts.
            \ucinfo{The system provides the user with the list of resolution options.
              \begin{itemize}
                \item By moving the overlapping event to another time slot.
                \item Keep both events with a conflict warning.
              \end{itemize}}
    \end{enumerate}
  }
  \ucalternateflows{
    \begin{enumerate}
      \item {No conflicts detected}
    \end{enumerate}
  }
  \ucexceptions{
    \begin{itemize}
      \item If the system doesn't get any possible way to resolve the conflict then the system would mark both the event as conflicting.
    \end{itemize}
  }
  \ucconclusion{The UC ends when the user chooses resolution weather if to reschedule the event or leaving it without resolving and it is reflected in on the calendar.}
  \ucpostconditions{The conflicting events in the calendar are either resolved or marked as conflicting}
  \ucspecialrequirements{The system give feasible conflict resolution options}
\end{usecase}

\begin{figure}[!h]
  \centering
  \includegraphics[width=\textwidth]{images/docs/diagrams/sequence-diagrams/all-sequence-diagrams/Suggest Conflict Resolutions.png}
  \caption{Suggest Conflict Resolutions Sequence Diagram}
  \label{fig:seq/suggest-conflict-resolutions}
\end{figure}

The sequence diagram in Figure~\ref{fig:seq/suggest-conflict-resolutions} illustrates the conflict detection and resolution workflow in Jadwal. When events are added to the calendar (either manually or through WhatsApp extraction), the System initiates a conflict check in the Database. This check specifically looks for temporal overlaps between events and identifies potential alternative time slots.

If conflicts are detected, the System follows a structured resolution process:
\begin{enumerate}
  \item Retrieves resolution options from the Database, which include:
        \begin{itemize}
          \item Moving the overlapping event to an alternative time slot
          \item Keeping both events with an explicit conflict warning
        \end{itemize}
  \item Obtains the device IDs associated with the customer who owns the conflicting events
  \item Utilizes Apple Push Notification service (APNs) to notify users about the conflict and present them with resolution options
\end{enumerate}

If no conflicts are detected, the System continues without any additional actions. This approach ensures users are promptly informed of scheduling conflicts while maintaining the flexibility to either reschedule events or knowingly maintain overlapping appointments. The workflow aligns with Jadwal's goal of providing straightforward conflict management while respecting user preferences in calendar organization.

This process specifically implements the use case requirements, focusing on practical conflict resolution through user notification and simple resolution options, rather than attempting automated resolution or complex prioritization schemes.
\begin{usecase}{Manage Scheduling Conflicts}
  \ucbasicinfo{Medium}{Regular}
  \ucshortdescription{This UC allows the user to manage scheduling conflicts by suggesting resolutions when overlapping events are detected.}
  \uctrigger{This UC is triggered when an automatically added event overlaps with an existing event.}
  \ucactors{User}{None}
  \ucpreconditions{
    \begin{itemize}
      \item User is logged in.
      \item Conflicting events list is not empty.
    \end{itemize}
  }
  \ucrelationships{N/A}{N/A}{N/A}
  \ucinputsoutputs{
    \begin{itemize}
      \item \textbf{Conflicting events} (Source: System)
      \item \textbf{Event to override} (Source: User)
    \end{itemize}
  }{
    \begin{itemize}
      \item \textbf{Conflict resolution suggestion} (Destination: User Interface)
      \item \textbf{Updated event schedules} (Destination: Calendar)
    \end{itemize}
  }
  \ucmainflow{
    \begin{enumerate}
      \item The user opens the application and clicks on the view conflicts icon
        \ucinfo{The application shows all the conflics with their resolution options}
      \item The user chooses the best fit option to manage each conflict 
        \ucinfo{The conflict is resolved and is removed from the conflict list }

    \end{enumerate}
  }
  \ucalternateflows{
    \begin{enumerate}
      \item If the user doesn't choose any option, it shows conflicting status until the user chooses any option or the event expires. 
      \item If the user clicks on reject the event is left overlapping.
    \end{enumerate}
  }
  \ucexceptions{
    \begin{itemize}
      \item Netwok failure 
    \end{itemize}
  }
  \ucconclusion{The UC ends when the conflicting events are either resolved or marked as conflicting, based on the user's choice.}
  \ucpostconditions{The calendar reflects the user's decision regarding event conflicts.}
  \ucspecialrequirements{The system should provide best suggestions for resolving conflicts.}
\end{usecase}

\begin{usecase}{Add Event Manually}
  \ucbasicinfo{High}{Regular}
  \ucshortdescription{This UC allows users to add events manually.}
  \uctrigger{The user clicks add event manually icon or a date on the calendar and adds the events.}
  \ucactors{User}{None}
  \ucpreconditions{The user is logged into the application.}
  \ucrelationships{Suggest Conflict Resolutions}{N/A}{N/A}
  \ucinputsoutputs{
    \begin{itemize}
      \item \textbf{Event Name} (Source: User)
      \item \textbf{Event Location} (Source: User)
      \item \textbf{Is all day?} (Source: User)
      \item \textbf{Event Date (Start and End)} (Source: User)
      \item \textbf{Event Time (Start and End)} (Source: User)
      \item \textbf{Event Description} (Source: User)
      \item \textbf{Notifications/Reminders} (Source: User)
    \end{itemize}
  }{
    \begin{itemize}
      \item \textbf{New Calendar event}
            (Destination: Calendar)
    \end{itemize}
  }
  \ucmainflow{
    \begin{enumerate}
      \item The user clicks the add event manually icon or a date on the calendar.
            \ucinfo{The add event manually form is displayed.}
      \item The user sets the details of the event in the respective fields and saves the event.
            \ucinfo{The event is displayed on the calendar with its details.}
    \end{enumerate}
  }
  \ucalternateflows{
    \begin{enumerate}
      \item If the validation fails the user can try again afte fixing the issues
    \end{enumerate}
  }
  \ucexceptions{
    \begin{itemize}
      \item The end time is before the start time.
      \item The user attempts to save the event without filling in mandatory fields.
    \end{itemize}
  }
  \ucconclusion{The UC ends when the event has been successfully added to the calendar, and displayed.}
  \ucpostconditions{The event is successfully added to the calendar and displayed in the correct time slot.}
  \ucspecialrequirements{The interface must be simple and allowing users to input events with less efforts.}
\end{usecase}

\begin{figure}[!h]
  \centering
  \includegraphics[width=\textwidth]{images/docs/diagrams/sequence-diagrams/all-sequence-diagrams/Add Event Manually.png}
  \caption{Add Event Manually Sequence Diagram}
  \label{fig:seq/add-event-manually}
\end{figure}

The "Add Event Manually Sequence Diagram", shown in \textbf{Figure~\ref{fig:seq/add-event-manually}}, illustrates the process flow when a user manually creates a new calendar event. The sequence begins when the user submits event details through the CreateEvent endpoint, triggering a series of validation and storage operations.

The Backend first performs comprehensive validation of the event details, checking for:
\begin{itemize}
  \item Mandatory fields completion (event name, date, time)
  \item Temporal logic (end time after start time)
  \item Format validity of all provided fields
\end{itemize}

Upon successful validation, the system executes the following steps:
\begin{enumerate}
  \item Stores the validated event in the Database
  \item Performs an automatic conflict check with existing events
  \item If conflicts are detected:
        \begin{itemize}
          \item Retrieves the device IDs associated with the event owner
          \item Dispatches a "New Conflict Detected" notification via Apple Push Notification service (APNs)
        \end{itemize}
  \item Returns an EventCreated response to the user
\end{enumerate}

If validation fails, the system immediately returns a ValidationError to the user, preventing invalid data from entering the system. This workflow ensures data integrity while providing immediate feedback about potential scheduling conflicts, maintaining calendar consistency and user awareness of overlapping appointments.

\subsection{Prayer Time and Notification Management}
Prayer time scheduling is a unique feature of Jadwal, and timely notifications ensure users never miss important events. These use cases detail how prayer times are scheduled and how the notification system keeps users informed.

\begin{usecase}{Schedule Prayer Times}
    \ucbasicinfo{High}{Regular}
    \ucshortdescription{The calendar is blocked and updated automatically according to the person's time zone prayer time.}
    \uctrigger{This usecase triggered when the user enables the prayer time feature in the application.}
    \ucactors{User}{None}
    \ucpreconditions{User must be logged into the system.}
    \ucrelationships{N/A}{N/A}{N/A}
    \ucinputsoutputs{
      \begin{itemize}
        \item \textbf{User's time Zone} (Source: user)
        \item \textbf{We get the IP address of the user and get their time zone}
      \end{itemize}
    }{
      \begin{itemize}
        \item \textbf{The calendar displays the blocked time for prayer time according to their time zone.}
      \end{itemize}
      }
    \ucmainflow{
      \begin{enumerate}
        \item User enables the feature by clicking the the option Schedule Prayer Time.
          \ucinfo{The system checks the time zone of the user and blocks the calendar accordingly.}
      \end{enumerate}
    }
    \ucalternateflows{
      \begin{itemize}
        \item The user doesn't enable the scheduling prayer time.
      \end{itemize}
    }
    \ucexceptions{
      \begin{itemize}
        \item If there's a system error, display a relevant error message.
      \end{itemize}
    }
    \ucpostconditions{The system generates calendar with prayer time}
    \ucspecialrequirements{The system must block the calendar according to their time zone}
    \ucconclusion{User's prayer times are successfully scheduled.}
    \ucbusinessrules{
      \begin{itemize}
        \item Prayer times must be within valid time ranges.
      \end{itemize}
    }
\end{usecase}
\begin{usecase}{Receive Event Notifications}
  \ucbasicinfo{High}{Regular}
  \ucshortdescription{Users receive notifications about upcoming events.}
  \uctrigger{The UC is triggered when the choosen time of an event's reminder has arrived}
  \ucactors{User}{None}
  \ucpreconditions{User must be logged into the system and set a reminder of the specific event.}
  \ucrelationships{N/A}{N/A}{N/A}
  \ucinputsoutputs{
    \begin{itemize}
      \item \textbf{Time of the event reminder} (Source: User)
    \end{itemize}
  }{
    \begin{itemize}

      \item \textbf{Notifications sent to users.} (Destination: System)
    \end{itemize}
  }
  \ucmainflow{
    \begin{enumerate}
      \item The system checks the alarms set for every event.
            \ucinfo{The system checks every 1 minute for set alarms for every event.}
      \item Notifications sent to users.
            \ucinfo{The user is reminded about the event by sending the notification}
    \end{enumerate}
  }
  \ucalternateflows{
    \begin{itemize}
      \item \textbf{If user denys the access to the notification the notifications are not sent.}
    \end{itemize}
  }
  \ucexceptions{
    \begin{itemize}
      \item \textbf{Network issue}
    \end{itemize}
  }
  \ucconclusion{The system checks for set alarm every 1 minute and if the event is detected the system sends the notification.}
  \ucpostconditions{Notifications are sent.}
  \ucspecialrequirements{Notification permission.}
  \ucbusinessrules{The system has to check every minute for the set alarm.
  }
\end{usecase}

\begin{figure}[!h]
  \centering
  \includegraphics[width=\textwidth]{images/docs/diagrams/sequence-diagrams/all-sequence-diagrams/Receive Event Notifications.png}
  \caption{Receive Event Notifications Sequence Diagram}
  \label{fig:seq/receive-event-notifications}
\end{figure}

The ``Receive Event Notifications Sequence Diagram'', shown in \textbf{Figure~\ref{fig:seq/receive-event-notifications}}, illustrates Jadwal's continuous event notification monitoring and delivery system. The sequence operates in a continuous polling loop that executes every minute, ensuring timely notification delivery for all scheduled events.

The polling process consists of several key steps:
\begin{enumerate}
  \item The System queries the Database for active alarms through regular checks
  \item For each batch of active alarms found:
        \begin{itemize}
          \item Retrieves associated device IDs for notification targets
          \item Iterates through each event requiring notification
          \item Dispatches push notifications via Apple Push Notification service (APNs)
        \end{itemize}
  \item If no active alarms are found, the system continues its polling cycle
\end{enumerate}

This robust notification system ensures reliable delivery of event reminders while efficiently managing system resources. The one-minute polling interval provides a balance between timely notifications and system performance, while the batch processing of notifications optimizes the interaction with APNs. The system's ability to handle multiple device IDs per user ensures notifications reach users across all their registered devices.

% ==========

\section{Activity Diagram}

\begin{figure}[!h]
    \centering
    \includegraphics[width=0.9\textwidth]{images/activity-diagram.png}
    \caption{Activity Diagram of Jadwal}
    \label{fig:activity-diagram}
\end{figure}

\newpage

An activity diagram is shown in Figure \ref{fig:activity-diagram} above, and it illustartes the flows the user of the app can take. It starts with the authentication whcih supports both Google and Email. Then afater the token is verified, a check is done to see if the user has a calendar or not. If the user has a calendar, we just move on, but if they don't have one, we create one and then move on. Here the user is inside the app and has two options, he can either view the calendar or open the settings page. When he views the calendar, he can view an event, or add one. Adding event might result in a conflict, so if one arises, they can try editing it and adding it again til no conflicts arise. After any of those two steps, the user can go back to view calendar or open settings page. If the choose to open settings page, they have three actions available. They can connect a CalDAV account, connect WhatsApp, or logout. Starting with the connecting a WhatsApp account, the system will extract events from the user's WhatsApp account automatically. If any conflicts arise, the system suggests a conflict resolution, and the user can take action by managing the scheduling conflict. But if no conflicts arise, the system just continues normally. Another option for the user is connecting a CalDAV account, and once that happens, the user can go back to having the option of going to calendar view or open settings page. Finally, the logout action and that terminates the session and the user is logged out of the app.



\section{Class Diagram}

The figures below show the main classes Jadwal's system includes. The figures give an overview of interfaces in the system to help in understanding the different parts of the system. Since Golang will be used for the backend mainly, the diagrams below are more suited for it.

Of course that doesn't mean they can't be used for implementing in other languages, but the way it was written took into consideration the language features. Golang is neither fully object oriented neither fully functional. It has \texttt{interface} and \texttt{struct}, and has implicit implementation. Implicit implementation means as long as a struct has the methods that an interface expects, that struct implemented the interface. Below is a key for the class diagrams coming below:
\begin{itemize}
    \item S - \texttt{struct} (Go struct type)
    \item T - \texttt{type} (Go type definition)
    \item C - \texttt{class} (Go interface implementations)
    \item I - \texttt{interface} (Go interface definition)
\end{itemize}

\newpage

\begin{figure}[!h]
    \centering
    \includegraphics[width=0.9\textwidth]{images/docs/diagrams/class/class-diagram/apimetadata.png}
    \caption{Api Metadata Class Diagram}
    \label{fig:api-metadata-class-diagram}
\end{figure}

In Figure \ref{fig:api-metadata-class-diagram}, the diagram shows the Api Metadata interface and its implementation along with important fields. It basically extracts data from the headers and includes it in the context of the requests. It helps you put stuff into the context and extract it from them. Context in Golang allows you to attach stuff to it, but it is not type safe, so we are making a wrapper to make it safe. Below we are adding wrappers for `Claims' and `Lang'. `Claims' are extracted from the JWT (JSON Web Token). `Lang' is self-explanatory.

\newpage

\begin{figure}[!h]
    \centering
    \includegraphics[width=0.9\textwidth]{images/docs/diagrams/class/class-diagram/auth.png}
    \caption{Auth Class Diagram}
    \label{fig:auth-class-diagram}
\end{figure}

In Figure \ref{fig:auth-class-diagram}, the diagram shows the Auth API interface with its dependencies. It also shows the \texttt{authv1connect} which is the generated code for the API. This includes the flow for using email to authenticate and using Google.

\newpage

\begin{figure}[!h]
    \centering
    \includegraphics[width=0.9\textwidth]{images/docs/diagrams/class/class-diagram/calendarv1.png}
    \caption{Calendar V1 Class Diagram}
    \label{fig:calendar-v1-class-diagram}
\end{figure}

In Figure \ref{fig:calendar-v1-class-diagram}, the diagram shows the api layer service calendar. It allows Jadwal's clients to manage their calendar data easily. It shows the interfaces and the implementation structs with their fields and methods.

\newpage

\begin{figure}[!h]
    \centering
    \includegraphics[width=0.9\textwidth]{images/docs/diagrams/class/class-diagram/googleclient.png}
    \caption{Google Client Class Diagram}
    \label{fig:googleclient-class-diagram}
\end{figure}

In Figure \ref{fig:googleclient-class-diagram}, the Google client, and its methods. It allows Jadawl to verify with Google that tokens its gets are good, and it also allows Jadwal system to get the user info the system was granted access for.

\newpage
    
\begin{figure}[!h]
    \centering
    \includegraphics[width=0.9\textwidth]{images/docs/diagrams/class/class-diagram/emailer.png}
    \caption{Emailer Class Diagram}
    \label{fig:emailer-class-diagram}
\end{figure}

In Figure \ref{fig:emailer-class-diagram}, the diagram shows the \texttt{emailer} interface with its three implementations. This allows for using the texttt{stdoutEmailer} when testing locally, and using either of the \texttt{resendEmailer} or \texttt{smtpEmailer} in production to send real emails.

\newpage
    
\begin{figure}[!h]
    \centering
    \includegraphics[width=0.9\textwidth]{images/docs/diagrams/class/class-diagram/store.png}
    \caption{Store Class Diagram}
    \label{fig:store-class-diagram}
\end{figure}

In Figure \ref{fig:store-class-diagram}, the diagram shows the store, or in other words the database. The store holds all the methods needed to query the database for information it holds. More methods can be added as needed, but those are the core methods.

\newpage
    
\begin{figure}[!h]
    \centering
    \includegraphics[width=0.9\textwidth]{images/docs/diagrams/class/class-diagram/tokens.png}
    \caption{Tokens Class Diagram}
    \label{fig:tokens-class-diagram}
\end{figure}

In Figure \ref{fig:tokens-class-diagram}, the diagram shows the tokens service implementation along with its interface. The diagram also illustartes the needed \texttt{struct} representation for any data passed to and from the service.

\newpage
    
\begin{figure}[!h]
    \centering
    \includegraphics[width=0.5\textwidth]{images/docs/diagrams/class/class-diagram/util.png}
    \caption{Util Class Diagram}
    \label{fig:util-class-diagram}
\end{figure}

In Figure \ref{fig:util-class-diagram}, the diagram shows config needed to support both production and development without committing secrets to the codebase. Falak in this context is the codename for the backend. The config here is represented as a \texttt{struct} since it is hold data and doesn't need implementations.

\section{Database Design}

The foundation of Jadwal's robust functionality lies in its carefully structured relational database architecture. This section presents both the Entity-Relationship model and Relational Schema that support the application's core features, from user authentication to calendar integration and event management. The database design ensures efficient data organization while maintaining the flexibility needed for future scalability.

\subsection{Entity-Relationship Model}

The Entity-Relationship model, illustrated in \textbf{Figure~\ref{fig:er-diagram}}, depicts the conceptual structure of Jadwal's database system. This model follows standard ER notation and demonstrates the relationships between the system's primary entities.

\subsubsection{Core Entities and Relationships}

\begin{enumerate}
    \item \textbf{User Management}
          \begin{itemize}
              \item \texttt{customer}: Central entity with attributes \texttt{id} (UUID), \texttt{name}, and \texttt{email}
              \item \texttt{auth\_google}: Represents Google OAuth authentication
              \item \texttt{magic\_link}: Manages email-based authentication tokens
          \end{itemize}

    \item \textbf{Calendar Organization}
          \begin{itemize}
              \item \texttt{calendar\_accounts}: Links customers to calendar providers
              \item \texttt{vcalendar}: Represents individual calendars with metadata
              \item One-to-many relationship from accounts to calendars
          \end{itemize}

    \item \textbf{Event Management}
          \begin{itemize}
              \item \texttt{vevent}: Core event entity with timing and details
              \item \texttt{vevent\_exception}: Handles recurring event modifications
              \item \texttt{valarm}: Manages event notifications
          \end{itemize}
\end{enumerate}

\subsubsection{Key Relationships}

\begin{itemize}
    \item Customer "has" authentication methods (one-to-many)
    \item Customer "owns" calendar accounts (one-to-many)
    \item Calendar accounts "contain" calendars (one-to-many)
    \item Calendars "schedule" events (one-to-many)
    \item Events "have" exceptions and alarms (one-to-many)
\end{itemize}

\subsection{Relational Database Schema}

The relational database schema, shown in \textbf{Figure~\ref{fig:relational-schema}}, provides the detailed logical structure of the database implementation. This representation demonstrates the actual table structures, attributes, and relationships as implemented in the database system.

\subsubsection{Table Structures}

\begin{enumerate}
    \item \textbf{Authentication Tables}
          \begin{itemize}
              \item \texttt{customer} (\underline{id}, name, email, created\_at, updated\_at)
              \item \texttt{auth\_google} (\underline{id}, customer\_id, sub, created\_at, updated\_at)
              \item \texttt{magic\_link} (\underline{id}, customer\_id, token\_hash, expires\_at, used\_at, created\_at, updated\_at)
          \end{itemize}

    \item \textbf{Calendar Tables}
          \begin{itemize}
              \item \texttt{calendar\_accounts} (\underline{id}, customer\_id, provider, created\_at, updated\_at)
              \item \texttt{vcalendar} (\underline{uid}, account\_id, prodid, version, display\_name, description, color, timezone, created\_at, updated\_at)
          \end{itemize}

    \item \textbf{Event Tables}
          \begin{itemize}
              \item \texttt{vevent} (\underline{uid}, calendar\_uid, dtstamp, dtstart, dtend, duration, summary, location, status, classification, transp, rrule, rdate, exdate, sequence, created\_at, updated\_at)
              \item \texttt{vevent\_exception} (\underline{id}, event\_uid, recurrence\_id, summary, description, location, dtstart, dtend, status, created\_at, updated\_at)
              \item \texttt{valarm} (\underline{id}, event\_uid, action, trigger, description, summary, duration, repeat, attendees, created\_at, updated\_at)
          \end{itemize}
\end{enumerate}

\subsubsection{Implementation Details}

\begin{itemize}
    \item \textbf{Primary Keys}: Underlined attributes represent primary keys
    \item \textbf{Foreign Keys}: Attributes ending in "\_id" or "\_uid" represent foreign key relationships
    \item \textbf{Audit Fields}: All tables include created\_at and updated\_at timestamps
    \item \textbf{Data Types}:
          \begin{itemize}
              \item UUID for unique identifiers
              \item VARCHAR for variable-length strings
              \item TIMESTAMPTZ for timezone-aware timestamps
              \item JSONB for complex data structures (rdate, exdate, attendees)
          \end{itemize}
\end{itemize}

\begin{figure}[!h]
    \centering
    \includegraphics[width=0.9\textwidth]{images/docs/diagrams/er/database/Database Design.png}
    \caption{Entity-Relationship Diagram}
    \label{fig:er-diagram}
\end{figure}

\begin{figure}[!h]
    \centering
    \includegraphics[width=\textwidth]{images/database-schema.png}
    \caption{Relational Schema}
    \label{fig:relational-schema}
\end{figure}

% ==========

\include{parts/ui-prototype-and-conlusion}

% ==========

\cleardoublepage
